\section{Safety Oracles}
Here, we present a minimum-viable safety oracle, and prove that it is sufficient to detect safety in some situations.

\subsection{Preliminary Definitions}

\begin{defn}[Later]
$$
Later:M \times M \to \{True, False\}
$$
$$
Later(m_1, m_2) :\Leftrightarrow m_2 \in Justification(m_1)
$$
We may use the infix operator:
$$
m_1 < m_2 :\Leftrightarrow m_2 \in Justification(m_1)
$$
% what's a better infix operator?
\end{defn}

\begin{defn}[From sender]
$$
From\_Sender: \mathcal{V} \times \mathcal{P}(M) \to \mathcal{P}(M)
$$
$$
From\_Sender(v, \sigma) = \{m \in \sigma : Sender(m) = v\}
$$
\end{defn}

\begin{defn}[Later messages from sender]
$$
Later\_From: M \times \mathcal{V} \times \mathcal{P}(M) \to \mathcal{P}(M)
$$
$$
Later\_From(m, v, \sigma) = \{m' \in \sigma : Sender(m') = v \land Later(m', m)\}
$$
\end{defn}

\begin{defn}[Latest messages]
$$
L_M:\Sigma \to (\mathcal{V} \to \mathcal{P}(M))
$$
$$
L_M(\sigma)(v) = \{m \in \sigma : Sender(m) = v ~\land~ Later\_From(m,v,\sigma) = \emptyset\}
$$
\end{defn}


\begin{defn}[Latest messages from non-Equivocating validators]
$$
L^H_M:\Sigma \to (\mathcal{V} \to \mathcal{P}(M))
$$
\[ L^H_M(\sigma)(v) = \left\{
\begin{array}{ll}
      \emptyset& \text{ for } v \in E(\sigma) \\
      L_M(\sigma)(v)& \text{ otherwise }
\end{array}
\right. \]
\end{defn}

Note that the map returned by this function has values $L^H_M(\sigma)(v) = \emptyset$ for any validators who are equivocating in $\sigma$ or who don't have any messages in $\sigma$.

\begin{defn}[Latest Estimates from non-Equivocating validators]
$$
L^H_E:\Sigma \to (\mathcal{V} \to \mathcal{P}(\mathcal{C}))
$$
$$
L^H_E(\sigma)(v) = \{Estimate(m) : m \in L^H_M(\sigma)(v)\}
$$
\end{defn}

As above, $L^H_E(\sigma)(v) = \emptyset$ for validators $v$ who are equivocating or missing in $\sigma$.


\begin{defn}[Latest justifications from non-Equivocating validators]
$$
L^H_J:\Sigma \to (\mathcal{V} \to \mathcal{P}(\Sigma))
$$
$$
L^H_J(\sigma)(v) = \{Justification(m) : m \in L^H_M(\sigma)(v)\}
$$
\end{defn}


The validators whose latest estimates agree with some proposition $p$ in some protocol state $\sigma$ is denoted formally as follows
\begin{defn}[Validators agreeing with $p$ in $\sigma$]
$$
Agreeing: P_{\mathcal{C}} \times \Sigma \to \mathcal{P}(\mathcal{V})
$$
$$
Agreeing(p, \sigma) :\Leftrightarrow \{v \in \mathcal{V} : \forall c \in L^H_E(\sigma)(v), p(c)\}
$$
\end{defn}

Similarly, we will define the set of validators validators whose latest estimates \emph{disagree} with $p$ in some protocol state $\sigma$:
\begin{defn}[Validators disagreeing with $p$ in $\sigma$]
$$
Disagreeing: P_{\mathcal{C}} \times \Sigma \to \mathcal{P}(\mathcal{V})
$$
$$
Disagreeing(p, \sigma) :\Leftrightarrow \{v \in \mathcal{V} : \exists c \in L^H_E(\sigma)(v), \neg p(c)\}
$$
\end{defn}

For convenience, we will derive a discrete weights measure from $W$, so that we can easily refer to the total weight of a set of validators.

\begin{defn}[Weight measure]
$$
Weight: \mathcal{P}(\mathcal{V}) \to \mathbb{R}_+
$$
$$
Weight(V) :\Leftrightarrow \sum_{v \in V} \mathcal{W}(v)
$$
\end{defn}


\begin{defn}[Non-equivocating majority driven property]
$$
Majority^H: \mathcal{P}(\mathcal{V}) \times \Sigma \to \{True, False\}
$$
$$
Majority^H(V, \sigma) :\Leftrightarrow Weight(V) > Weight(\mathcal{V} \setminus E(\sigma))/2
$$
\end{defn}

\begin{defn}[Non-equivocating majority driven property] % to be redefined in terms of Majority^H ?
$$
Majority\_Driven^H: P_{\mathcal{C}} \to \{True, False\}
$$
$$
Majority\_Driven^H(p) :\Leftrightarrow \forall \sigma \in \Sigma, Majority^H(Agreeing(p, \sigma), \sigma) \implies \forall c \in \mathcal{E}(\sigma), p(c)
$$
\end{defn}

\begin{defn}[Max weight driven property]
$$
Max\_Driven: P_{\mathcal{C}} \to \{True, False\}
$$
$$
Max\_Driven(p) :\Leftrightarrow \forall \sigma \in \Sigma, Weight(Agreeing(p, \sigma)) > Weight(Disagreeing(p, \sigma)) \implies \forall c \in \mathcal{E}(\sigma), p(c)
$$
\end{defn}


\begin{defn}[Later disagreeing messages]
$$
Later\_Disagreeing: P_{\mathcal{C}} \times M \times \mathcal{V} \times \Sigma \to \mathcal{P}(M)
$$
$$
Later\_Disagreeing(p,m,v,\sigma) = \{m' \in Later\_From(m, v, \sigma) : \neg p(Estimate(m')) \}
$$
\end{defn}

Since non-equivocating validators have not more than one latest message/estimate, we will define the following higher level helper function:

\begin{defn}
$$
!: \mathcal{P}(X) \to X
$$
$$
!(A) = a \in A : \forall b \in A, b = a
$$
\end{defn}

This function is not defined when its argument is not singleton (i.e. when is multi-element or the empty set).

\subsection{Validator Cliques}

A clique of validators for some proposition $p$ can be thought of as a set of validators who pairwise see eachother agreeing with $p$, where there are also no later messages from these validators that are disagreeing with $p$. A clique can have mulitple layers, although the minimum viable safety oracle will only be $1-layer$ deep. \\

\begin{defn}[Clique with $n$ layers]
\begin{align}
Clique&: \mathbb{N}^+ \times \mathcal{P}(\mathcal{V}) \times P_{\mathcal{C}} \times \Sigma \to \{True, False\} \\
Clique(1, V, p, \sigma) &:\Leftrightarrow \forall v \in V, V \subseteq Agreeing(p, L^H_J(\sigma)(v)) \\
&~~~~\land \forall v' \in V, Later\_Disagreeing(p, L^H_M(L^H_J(\sigma)(v))(v'), v', \sigma) = \emptyset \\
Clique(n, V, p, \sigma) &:\Leftrightarrow \forall v \in V, V \subseteq Agreeing(p, L^H_J(\sigma)(v)) \\
&~~~~\land \forall v \in V, Clique\_Cond(n - 1, V, p, L^H_J(\sigma)(v))
\end{align}
\end{defn}


\subsection{Cliques Survive Messages from Validators Outside Clique}

First, we define minimal transitions:

\begin{defn}[Minimal transitions]
$$
Minimal_t = \{ (\sigma, \sigma') \in \Sigma_t^2: \sigma \to \sigma' \land \nexists \sigma'' . \sigma \to \sigma'' \to \sigma' \land \sigma'' \neq \sigma \land \sigma'' \neq \sigma' \}
$$
\end{defn}

A minimal transition corresponds to recieving a single new message with justification drawn from the initial protocol state. We will show that any minimal transition from a validator not in the clique will not affect the clique.

We now will show that minimal transitions don't affect later messages, equivocation status, latest messages, or later disagreeing messages of validators that are not the sender of the minimal transition.

\begin{lemma}[Minimal transitions do not change Later\_From for any non-sender]
$\forall (\sigma, \sigma') \in Minimal_t, \forall m \in M$, letting $\overline{m} =~!\sigma'\setminus\sigma$ and $\overline{v} = Sender(\overline{m})$,
$$
\forall v \in \mathcal{V} \setminus \{\overline{v}\}, Later\_From(m,v,\sigma) = Later\_From(m,v,\sigma')
$$
\end{lemma}


\begin{proof}
$\forall v \in \mathcal{V} \setminus \{\overline{v}\}$
\begin{align}
 &Later\_From(m,v,\sigma) \\
=&\{m' \in \sigma : Sender(m') = v \land Later(m', m)\} \\
=&\{m' \in \sigma : Sender(m') = v \land Later(m', m)\} \cup \emptyset \\
=&\{m' \in \sigma : Sender(m') = v \land Later(m', m)\} \cup \{m' \in \{\overline{m}\}: Sender(m') = v\} \\
=&\{m' \in \sigma : Sender(m') = v \land Later(m', m)\} \cup \{m' \in \{\overline{m}\}: Sender(m') = v \land Later(m', m)\} \\
=&\{m' \in \sigma \cup \{\overline{m}\} : Sender(m') = v \land Later(m', m)\} \\
=&\{m' \in \sigma' : Sender(m') = v \land Later(m', m)\} \\
=&Later\_From(m,v,\sigma')
\end{align}
\end{proof}

\begin{lemma}[Minimal transitions do not change equivocation status for any non-sender]
$\forall (\sigma, \sigma') \in Minimal_t$, letting $\overline{m} =~!\sigma'\setminus\sigma$, and $\overline{v} = Sender(\overline{m})$,
$$
\forall v \in \mathcal{V} \setminus \{\overline{v}\}, (v \in E(\sigma) \iff v \in E(\sigma'))
$$
\end{lemma}

\begin{proof}
The forward direction, $v \in E(\sigma) \implies v \in E(\sigma')$, follows immediatly as equivocation is monotonic,
\begin{align}
        &v \in E(\sigma) \\
\implies&v \in E(\sigma) \land \sigma \to \sigma' \\
\implies&v \in E(\sigma) \land \sigma \subseteq \sigma' \\
\implies&v \in E(\sigma')&
\end{align}


The backwards direction, $v \in E(\sigma') \implies v \in E(\sigma)$,
\begin{align}
        &v \in E(\sigma') \\
\implies& v \in \{v' \in \mathcal{V}: \exists m_1 \in \sigma', \exists m_2 \in \sigma', m_1 \perp m_2 \land Sender(m_1) = v' \} \\
\implies& \exists m_1 \in \sigma', \exists m_2 \in \sigma', m_1 \perp m_2 \land Sender(m_1) = v \\
\implies& \exists m_1 \in \sigma', \exists m_2 \in \sigma', m_1 \perp m_2 \land Sender(m_1) = v \land Sender(\overline{m}) \neq v \\
\implies& \exists m_1 \in \sigma \cup \{\overline{m}\}, \exists m_2 \in \sigma', m_1 \perp m_2 \land Sender(m_1) = v \land Sender(\overline{m}) \neq v \\
\implies& \exists m_1 \in \sigma, \exists m_2 \in \sigma', m_1 \perp m_2 \land Sender(m_1) = v \land Sender(\overline{m}) \neq v \\
        &\lor (\exists m_2 \in \sigma', \overline{m} \perp m_2 \land Sender(\overline{m}) = v \land Sender(\overline{m}) \neq v) \\
\implies& \exists m_1 \in \sigma, \exists m_2 \in \sigma', m_1 \perp m_2 \land Sender(m_1) = v \land Sender(\overline{m}) \neq v \lor False \\
\implies& \exists m_1 \in \sigma, \exists m_2 \in \sigma', m_1 \perp m_2 \land Sender(m_1) = v \land Sender(\overline{m}) \neq v \\
\implies& \exists m_1 \in \sigma, \exists m_2 \in \sigma \cup \{\overline{m}\}, m_1 \perp m_2 \land Sender(m_1) = v \land Sender(\overline{m}) \neq v \\
\implies& \exists m_1 \in \sigma, (\exists m_2 \in \sigma, m_1 \perp m_2 \land Sender(m_1) = v \land Sender(\overline{m}) \neq v \\
        &\lor (m_1 \perp \overline{m} \land Sender(m_1) = v \land Sender(\overline{m}) \neq v) \\
\iff& \exists m_1 \in \sigma, (\exists m_2 \in \sigma, m_1 \perp m_2 \land Sender(m_1) = v \land Sender(\overline{m}) \neq v\\
        &\lor (m_1 \perp \overline{m} \land Sender(m_1) = v \land Sender(m_1) = Sender(\overline{m}) \land Sender(\overline{m}) \neq v) \\
\iff& \exists m_1 \in \sigma, \exists m_2 \in \sigma, m_1 \perp m_2 \land Sender(m_1) = v \land Sender(\overline{m}) \neq v \lor False \\
\implies& \exists m_1 \in \sigma, \exists m_2 \in \sigma, m_1 \perp m_2 \land Sender(m_1) = v \\
\implies& v \in \{ v' \in \mathcal{V} : \exists m_1 \in \sigma, \exists m_2 \in \sigma, m_1 \perp m_2 \land Sender(m_1) = v' \} \\
\implies& v \in E(\sigma)
\end{align}
\end{proof}


\begin{lemma}[Minimal transitions do not change latest messages for any non-sender]
$\forall (\sigma, \sigma') \in Minimal_t$, letting $\overline{m} =~!\sigma'\setminus\sigma$, and $\overline{v} = Sender(\overline{m})$
$$
\forall v \in \mathcal{V} \setminus \{\overline{v}\}, L^H_M(\sigma)(v) = L^H_M(\sigma')(v)
$$
\end{lemma}

\begin{proof}
Due to the definition of $L^H_M$, there are two cases: when $v \in E(\sigma)$ and otherwise:

\[ L^H_M(\sigma))(v) = \left\{
\begin{array}{ll}
      \emptyset& \text{ for } v \in E(\sigma) \\
      L_M(\sigma)(v) \text{ otherwise }
\end{array}
\right.\]

Case 1, $\forall v \in \mathcal{V} \setminus \{\overline{v}\} : v \in E(\sigma)$:

\begin{align}
&L^H_M(\sigma)(v) = \emptyset \\
\implies&L^H_M(\sigma)(v) = \emptyset \land (v \in E(\sigma) \iff v \in E(\sigma')) \\
\implies&L^H_M(\sigma)(v) = \emptyset \land v \in E(\sigma') \\
\implies&L^H_M(\sigma)(v) = \emptyset \land L^H_M(\sigma')(v) = \emptyset \\
\implies&L^H_M(\sigma)(v) = L^H_M(\sigma')(v) \\
\end{align}

Case 2, $\forall v \in \mathcal{V} \setminus \{\overline{v}\} : v \notin E(\sigma)$:

\begin{align}
        &L^H_M(\sigma)(v) = L_M(\sigma)(v) \\
\implies&L^H_M(\sigma)(v) = \{m \in \sigma : Sender(m) = v ~\land~ Later\_From(m,v,\sigma) = \emptyset\} \\
\implies&L^H_M(\sigma)(v) = \{m \in \sigma : Sender(m) = v ~\land~ Later\_From(m,v,\sigma) = \emptyset\} \\
        &\land \forall m^* \in M, Later\_From(m^*,v,\sigma) = Later\_From(m^*,v,\sigma')\\
\implies&L^H_M(\sigma)(v) = \{m \in \sigma : Sender(m) = v ~\land~ Later\_From(m,v,\sigma') = \emptyset\} \\
\implies&L^H_M(\sigma)(v) = \{m \in \sigma : Sender(m) = v ~\land~ Later\_From(m,v,\sigma') = \emptyset\} \\
        &\cup \emptyset \\
\implies&L^H_M(\sigma)(v) = \{m \in \sigma : Sender(m) = v ~\land~ Later\_From(m,v,\sigma') = \emptyset\} \\
        &\cup \{m \in \{\overline{m}\} : Sender(m) = v\} \\
\implies&L^H_M(\sigma)(v) = \{m \in \sigma : Sender(m) = v ~\land~ Later\_From(m,v,\sigma') = \emptyset\} \\
        &\cup \{m \in \{\overline{m}\} : Sender(m) = v ~\land~ Later\_From(m,v,\sigma') = \emptyset\} \\
\implies&L^H_M(\sigma)(v) = \{m \in \sigma \cup \{\overline{m}\}: Sender(m) = v ~\land~ Later\_From(m,v,\sigma') = \emptyset\} \\
\implies&L^H_M(\sigma)(v) = \{m \in \sigma': Sender(m) = v ~\land~ Later\_From(m,v,\sigma') = \emptyset\} \\
\implies&L^H_M(\sigma)(v) = L^H_M(\sigma')(v)
\end{align}
\end{proof}

\begin{lemma}[Minimal transitions do not change latest justification for any non-sender]
$\forall (\sigma, \sigma') \in Minimal_t$, letting $\overline{m} =~!\sigma'\setminus\sigma$, and $\overline{v} = Sender(\overline{m})$
$$
\forall v \in \mathcal{V} \setminus \{\overline{v}\}, L^H_J(\sigma)(v) = L^H_J(\sigma')(v)
$$
\end{lemma}

\begin{proof}
  $\forall v \in \mathcal{V} \setminus \{\overline{v}\}$,
  \begin{align}
            &L^H_J(\sigma)(v) = \{Justification(m) : m \in L^H_M(\sigma)(v)\} \\
    \implies&L^H_J(\sigma)(v) = \{Justification(m) : m \in L^H_M(\sigma)(v)\} \\
            &\land \forall v \in \mathcal{V} \setminus \{\overline{v}\}, L^H_M(\sigma)(v) = L^H_M(\sigma')(v) \\
    \implies&L^H_J(\sigma)(v) = \{Justification(m) : m \in L^H_M(\sigma')(v)\} \\
    \implies&L^H_J(\sigma)(v) = L^H_J(\sigma')(v) \\
  \end{align}
\end{proof}



\begin{lemma}[Minimal transitions do not change Later\_Disagreeing for any non-sender]
$\forall p \in P_{\mathcal{C}}, \forall (\sigma, \sigma') \in Minimal_t, \forall m \in M$, letting $\overline{m} =~!\sigma'\setminus\sigma$ and $\overline{v} = Sender(\overline{m})$
$$
\forall v \in \mathcal{V} \setminus \{\overline{v}\}, Later\_Disagreeing(p, m, v, \sigma) = Later\_Disagreeing(p, m, v, \sigma')
$$
\end{lemma}

\begin{proof}
$\forall v \in \mathcal{V} \setminus \{\overline{v}\},$
\begin{align}
Later\_Disagreeing(p, m, v, \sigma) &= \{m' \in Later\_From(m, v, \sigma) : \neg p(Estimate(m')) \} \\
\implies Later\_Disagreeing(p, m, v, \sigma) &= \{m' \in Later\_From(m, v, \sigma) : \neg p(Estimate(m')) \} \\
~~\land \forall m^* \in M, Later\_From(m^*,v,\sigma) &= Later\_From(m^*,v,\sigma') \\
\implies Later\_Disagreeing(p, m, v, \sigma) &= \{m' \in Later\_From(m, v, \sigma') : \neg p(Estimate(m')) \} \\
\implies Later\_Disagreeing(p, m, v, \sigma) &= Later\_Disagreeing(p, m, v, \sigma') \\
\end{align}
\end{proof}

\begin{lemma}[Minimal transition from outside clique maintains clique]
$\forall p \in P_{\mathcal{C}}, \forall (\sigma, \sigma') \in Minimal_t, \forall V \subseteq \mathcal{V}$, letting $\overline{m} =~!\sigma'\setminus\sigma$ and $\overline{v} = Sender(\overline{m})$
$$
\overline{v} \notin V \land Clique(1, V, p, \sigma) \implies Clique(1, V, p, \sigma')
$$
\end{lemma}

\begin{proof}
  \begin{align}
    &\overline{v} \notin V \land Clique(1, V, p, \sigma) \\
    \implies& \overline{v} \notin V \land [\forall v \in V, V \subseteq Agreeing(p, L^H_J(\sigma)(v)) \\
            &~~~~\land \forall v' \in V, Later\_Disagreeing(p, L^H_M(L^H_J(\sigma)(v))(v'), v', \sigma) = \emptyset] \\
    \implies& \overline{v} \notin V \land [\forall v \in V, V \subseteq Agreeing(p, L^H_J(\sigma)(v)) \\
            &~~~~\land \forall v' \in V, Later\_Disagreeing(p, L^H_M(L^H_J(\sigma)(v))(v'), v', \sigma) = \emptyset] \\
            &\land \forall v \in \mathcal{V} \setminus \{\overline{v}\}, L^H_J(\sigma)(v) = L^H_J(\sigma')(v) \\
            &\land V \subseteq \mathcal{V} \setminus \{\overline{v}\} \\
    \implies& \overline{v} \notin V \land [\forall v \in V, V \subseteq Agreeing(p, L^H_J(\sigma)(v)) \\
            &~~~~\land \forall v' \in V, Later\_Disagreeing(p, L^H_M(L^H_J(\sigma)(v))(v'), v', \sigma) = \emptyset] \\
            &\land \forall v \in V, L^H_J(\sigma)(v) = L^H_J(\sigma')(v) \\
    \implies& \overline{v} \notin V \land [\forall v \in V, V \subseteq Agreeing(p, L^H_J(\sigma')(v)) \\
            &~~~~\land \forall v' \in V, Later\_Disagreeing(p, L^H_M(L^H_J(\sigma')(v))(v'), v', \sigma) = \emptyset] \\
    \implies& \overline{v} \notin V \land [\forall v \in V, V \subseteq Agreeing(p, L^H_J(\sigma')(v)) \\
            &~~~~\land \forall v' \in V, Later\_Disagreeing(p, L^H_M(L^H_J(\sigma')(v))(v'), v', \sigma) = \emptyset] \\
            &\land \forall v \in \mathcal{V} \setminus \{\overline{v}\}, \forall m \in M, Later\_Disagreeing(p, m, v, \sigma) = Later\_Disagreeing(p, m, v, \sigma') \\
            &\land V \subseteq \mathcal{V} \setminus \{\overline{v}\} \\
    \implies& \overline{v} \notin V \land [\forall v \in V, V \subseteq Agreeing(p, L^H_J(\sigma')(v)) \\
            &~~~~\land \forall v' \in V, Later\_Disagreeing(p, L^H_M(L^H_J(\sigma')(v))(v'), v', \sigma) = \emptyset] \\
            &\land \forall v \in V, \forall m \in M, Later\_Disagreeing(p, m, v, \sigma) = Later\_Disagreeing(p, m, v, \sigma') \\
    \implies& \overline{v} \notin V \land [\forall v \in V, V \subseteq Agreeing(p, L^H_J(\sigma')(v)) \\
            &~~~~\land \forall v' \in V, Later\_Disagreeing(p, L^H_M(L^H_J(\sigma')(v))(v'), v', \sigma') = \emptyset] \\
    \implies& Clique(1, V, p, \sigma') \\
  \end{align}
\end{proof}

%---------------------------------------------------------------------------------------------------------------------
%---------------------------------------------------------------------------------------------------------------------
%---------------------------------------------------------------------------------------------------------------------
%---------------------------------------------------------------------------------------------------------------------
%---------------------------------------------------------------------------------------------------------------------
%---------------------------------------------------------------------------------------------------------------------
%---------------------------------------------------------------------------------------------------------------------


\subsection{Majority Cliques Survive Honest Messages from Validators in Clique}

%---------------------------------------------------------------------------------------------------------------------
%---------------------------------------------------------------------------------------------------------------------
%---------------------------------------------------------------------------------------------------------------------


\subsubsection{New messages at least leaves a smaller clique behind}

\begin{lemma}[Free sub-clique]
$\forall p \in P_{\mathcal{C}}, \forall V \subseteq \mathcal{V}, \forall (\sigma, \sigma') \in Minimal_t$, letting $\overline{m} =~!\sigma'\setminus\sigma$ and $\overline{v} = Sender(\overline{m})$,

$$
Clique(1, V, p, \sigma) \implies Clique(1, V \setminus \{\overline{v}\}, p, \sigma')
$$
\end{lemma}


\begin{proof}
\begin{align}
&Clique(1, V, p, \sigma)  \\
\implies& \forall v \in V, V \subseteq Agreeing(p, !L^H_J(\sigma)(v)) \\
        &~~~~~ \land \forall v' \in V,Later\_Disagreeing(p,~!L^H_M(!L^H_J(\sigma)(v))(v'), v', \sigma) = \emptyset \\
\implies& \forall v \in V, V \subseteq Agreeing(p, !L^H_J(\sigma)(v)) \\
        &~~~~~ \land \forall v' \in V,Later\_Disagreeing(p,~!L^H_M(!L^H_J(\sigma)(v))(v'), v', \sigma) = \emptyset \\
        &\land \forall v \in \mathcal{V} \setminus \{\overline{v}\}, L^H_J(\sigma)(v) = L^H_J(\sigma')(v) \\
        &\land V \setminus \{\overline{v}\} \subseteq \mathcal{V} \setminus \{\overline{v}\} \\
\implies& \forall v \in V, V \subseteq Agreeing(p, !L^H_J(\sigma)(v)) \\
        &~~~~~ \land \forall v' \in V,Later\_Disagreeing(p,~!L^H_M(!L^H_J(\sigma)(v))(v'), v', \sigma) = \emptyset \\
        &\land \forall V \setminus \{\overline{v}\}, L^H_J(\sigma)(v) = L^H_J(\sigma')(v) \\
\implies& \forall v \in V \setminus \{\overline{v}\}, V \subseteq Agreeing(p, !L^H_J(\sigma')(v)) \\
        &~~~~~ \land \forall v' \in V \setminus \{\overline{v}\},Later\_Disagreeing(p,~!L^H_M(!L^H_J(\sigma')(v))(v'), v', \sigma) = \emptyset \\
\implies& \forall v \in V \setminus \{\overline{v}\}, V \subseteq Agreeing(p, !L^H_J(\sigma')(v)) \\
        &~~~~~ \land \forall v' \in V \setminus \{\overline{v}\},Later\_Disagreeing(p,~!L^H_M(!L^H_J(\sigma')(v))(v'), v', \sigma) = \emptyset \\
        &\land \forall v \in \mathcal{V} \setminus \{\overline{v}\}, \forall m \in M, Later\_Disagreeing(p, m, v, \sigma) = Later\_Disagreeing(p, m, v, \sigma') \\
        &\land V \setminus \{\overline{v}\} \subseteq \mathcal{V} \setminus \{\overline{v}\} \\
\implies& \forall v \in V \setminus \{\overline{v}\}, V \subseteq Agreeing(p, !L^H_J(\sigma')(v)) \\
        &~~~~~ \land \forall v' \in V \setminus \{\overline{v}\},Later\_Disagreeing(p,~!L^H_M(!L^H_J(\sigma')(v))(v'), v', \sigma) = \emptyset \\
        &\land \forall v \in V \setminus \{\overline{v}\}, \forall m \in M, Later\_Disagreeing(p, m, v, \sigma) = Later\_Disagreeing(p, m, v, \sigma') \\
\implies& \forall v \in V \setminus \{\overline{v}\}, V \subseteq Agreeing(p, !L^H_J(\sigma')(v)) \\
        &~~~~~ \land \forall v' \in V \setminus \{\overline{v}\},Later\_Disagreeing(p,~!L^H_M(!L^H_J(\sigma')(v))(v'), v', \sigma') = \emptyset \\
\implies& \forall v \in V \setminus \{\overline{v}\}, V \subseteq Agreeing(p, !L^H_J(\sigma')(v)) \\
        &~~~~~ \land \forall v' \in V \setminus \{\overline{v}\},Later\_Disagreeing(p,~!L^H_M(!L^H_J(\sigma')(v))(v'), v', \sigma') = \emptyset \\
        &\land V \setminus \{\overline{v}\} \subseteq V \\
\implies& \forall v \in V \setminus \{\overline{v}\}, V \setminus \{\overline{v}\} \subseteq Agreeing(p, !L^H_J(\sigma')(v)) \\
        &~~~~~ \land \forall v' \in V \setminus \{\overline{v}\},Later\_Disagreeing(p,~!L^H_M(!L^H_J(\sigma')(v))(v'), v', \sigma') = \emptyset \\
\implies& Clique(1, V \setminus \{\overline{v}\}, p, \sigma') \\
\end{align}
\end{proof}


\subsubsection{Non-equivocating messages from clique members see clique agree}
\begin{lemma}[Later messages from a non-equivocating validator include all earlier messages]
$\forall \sigma \in \Sigma$,
\begin{align}
        &v \notin E(\sigma) \land \sigma_1 \in \Sigma ~\land \sigma_1 \subseteq \sigma \land \sigma_2 \subseteq \sigma \land \sigma_1 \cap \sigma_2 = \emptyset \\
\implies& \forall m_1 \in \sigma_1 : Sender(m_1) = v, \forall m_2 \in \sigma_2 : Sender(m_2) = v, m_1 \in Justification(m_2)
\end{align}
Note that $\sigma_2$ is not a protocol state, but just a collection of messages. The notation should probably be changed!
\end{lemma}



\begin{proof}
  \begin{align}
          &v \notin E(\sigma) \\
  \implies&v \notin \{v' \in \mathcal{V} : \exists m_1 \in \sigma, \exists m_2 \in \sigma, ~ m_1 \perp m_2 ~ \land ~ Sender(m_1) = v' \} \\
  \implies&\neg (\exists m_1 \in \sigma, \exists m_2 \in \sigma, ~ m_1 \perp m_2 ~ \land ~ Sender(m_1) = v) \\
  \implies&\forall m_1 \in \sigma, \forall m_2 \in \sigma, ~ m_1 \not\perp m_2 ~ \lor ~ Sender(m_1) \neq v \\
  \end{align}
  Thus, if we are only considering messages from $v$, they must not be equivocations.

  \begin{align}
          &\forall m_1 \in \sigma: Sender(m_1) = v, \forall m_2 \in \sigma: Sender(m_2) = v, ~ m_1 \not\perp m_2 \\
  \implies&\forall m_1 \in \sigma: Sender(m_1) = v, \forall m_2 \in \sigma: Sender(m_2) = v, ~ m_1 \not\perp m_2 \land \sigma_1 \subseteq \sigma \land \sigma_2 \subseteq \sigma \\
  \implies&\forall m_1 \in \sigma_1: Sender(m_1) = v, \forall m_2 \in \sigma_2: Sender(m_2) = v, ~ m_1 \not\perp m_2 \\
  \implies&\forall m_1 \in \sigma_1: Sender(m_1) = v, \forall m_2 \in \sigma_2: Sender(m_2) = v, \\
          &\neg (Sender(m_1) = Sender(m_2) \land m_1 \neq m_2 \land m_1 \notin Justification(m_2) \land m_2 \notin Justification(m_1)) \\
  \implies&\forall m_1 \in \sigma_1: Sender(m_1) = v, \forall m_2 \in \sigma_2: Sender(m_2) = v, \\
          &Sender(m_1) \neq Sender(m_2) \lor m_1 = m_2 \lor m_1 \in Justification(m_2) \lor m_2 \in Justification(m_1) \\
  \implies&\forall m_1 \in \sigma_1: Sender(m_1) = v, \forall m_2 \in \sigma_2: Sender(m_2) = v, \\
          &Sender(m_1) \neq Sender(m_2) \lor m_1 = m_2 \lor m_1 \in Justification(m_2) \lor m_2 \in Justification(m_1) \\
          &\land Sender(m_1) = v \land Sender(m_2) = v \\
  \implies&\forall m_1 \in \sigma_1: Sender(m_1) = v, \forall m_2 \in \sigma_2: Sender(m_2) = v, \\
          &m_1 = m_2 \lor m_1 \in Justification(m_2) \lor m_2 \in Justification(m_1) \\
  \implies&\forall m_1 \in \sigma_1: Sender(m_1) = v, \forall m_2 \in \sigma_2: Sender(m_2) = v, \\
          &m_1 = m_2 \lor m_1 \in Justification(m_2) \lor m_2 \in Justification(m_1) \\
          &\land m_1 \in \sigma_1 \land m_2 \notin \sigma_1 \\
  \implies&\forall m_1 \in \sigma_1: Sender(m_1) = v, \forall m_2 \in \sigma_2: Sender(m_2) = v, \\
          &m_1 = m_2 \lor m_1 \in Justification(m_2) \lor m_2 \in Justification(m_1) \\
          &\land m_1 \neq m_2 \\
  \implies&\forall m_1 \in \sigma_1: Sender(m_1) = v, \forall m_2 \in \sigma_2: Sender(m_2) = v, \\
          &m_1 \in Justification(m_2) \lor m_2 \in Justification(m_1) \\
  \implies&\forall m_1 \in \sigma_1: Sender(m_1) = v, \forall m_2 \in \sigma_2: Sender(m_2) = v, \\
          &m_1 \in Justification(m_2) \lor m_2 \in Justification(m_1) \\
          &\land \forall \sigma^* \in \Sigma, \forall m^* \in \sigma^*, Justification(m^*) \subseteq \sigma^* \\
  \implies&\forall m_1 \in \sigma_1: Sender(m_1) = v, \forall m_2 \in \sigma_2: Sender(m_2) = v, \\
          &m_1 \in Justification(m_2) \lor m_2 \in Justification(m_1) \\
          &\land \forall m^* \in \sigma_1, Justification(m^*) \subseteq \sigma_1 \\
  \implies&\forall m_1 \in \sigma_1: Sender(m_1) = v, \forall m_2 \in \sigma_2: Sender(m_2) = v, \\
          &m_1 \in Justification(m_2) \lor m_2 \in Justification(m_1) \\
          &\land \forall m^* \in \sigma_1, (m' \in Justification(m^*) \implies m' \in \sigma_1) \\
  \implies&\forall m_1 \in \sigma_1: Sender(m_1) = v, \forall m_2 \in \sigma_2: Sender(m_2) = v, \\
          &m_1 \in Justification(m_2) \lor m_2 \in Justification(m_1) \\
          &\land (m_2 \in Justification(m_1) \implies m_2 \in \sigma_1) \\
  \implies&\forall m_1 \in \sigma_1: Sender(m_1) = v, \forall m_2 \in \sigma_2: Sender(m_2) = v, \\
          &m_1 \in Justification(m_2) \lor m_2 \in Justification(m_1) \\
          &\land (m_2 \notin \sigma_1 \implies m_2 \notin Justification(m_1)) \land m_2 \notin \sigma_1 \\
  \implies&\forall m_1 \in \sigma_1: Sender(m_1) = v, \forall m_2 \in \sigma_2: Sender(m_2) = v, \\
          &m_1 \in Justification(m_2) \lor m_2 \in Justification(m_1) \land m_2 \notin Justification(m_1) \\
  \implies&\forall m_1 \in \sigma_1: Sender(m_1) = v, \forall m_2 \in \sigma_2: Sender(m_2) = v, m_1 \in Justification(m_2)\\
  \end{align}
\end{proof}

\iffalse
This is a specific version of the above proof. It's here because it existed first.
\begin{lemma}[$\overline{m}$ is $\overline{v}$'s latest message in $\sigma$']
$\forall (\sigma, \sigma') \in Minimal_t$, letting $\overline{m} =~!\sigma'\setminus\sigma$ and $\overline{v} = Sender(\overline{m})$,
$$
\overline{v} \notin E(\sigma') \implies \forall m \in \sigma: Sender(m) = \overline{v}, m \in Justification(\overline{m})
$$
\end{lemma}

\begin{proof}
\begin{align}
        &\overline{v} \notin E(\sigma') \\
\implies&\overline{v} \notin \{v \in \mathcal{V} : \exists m_1 \in \sigma', \exists m_2 \in \sigma', ~ m_1 \perp m_2 ~ \land ~ Sender(m_1) = v \} \\
\implies&\neg (\exists m_1 \in \sigma', \exists m_2 \in \sigma', ~ m_1 \perp m_2 ~ \land ~ Sender(m_1) = \overline{v}) \\
\implies&\forall m_1 \in \sigma', \forall m_2 \in \sigma', ~ m_1 \not\perp m_2 ~ \lor ~ Sender(m_1) \neq \overline{v} \\
\end{align}
Thus, if we are only considering messages from $\overline{v}$, they must not be equivocations.
\begin{align}
        &\forall m_1 \in \sigma': Sender(m_1) = \overline{v}, \forall m_2 \in \sigma': Sender(m_2) = \overline{v}, ~ m_1 \not\perp m_2 \\
\implies&\forall m_1 \in \sigma': Sender(m_1) = \overline{v}, \forall m_2 \in \sigma \cup \{\overline{m}\}: Sender(m_2) = \overline{v}, ~ m_1 \not\perp m_2 \\
\implies&\forall m_1 \in \sigma': Sender(m_1) = \overline{v}, \forall m_2 \in \{\overline{m}\}: Sender(m_2) = \overline{v}, ~ m_1 \not\perp m_2 \\
\implies&\forall m_1 \in \sigma': Sender(m_1) = \overline{v}, \forall m_2 \in \{\overline{m}\}: Sender(m_2) = \overline{v}, ~ m_1 \not\perp m_2 \land Sender(\overline{m}) = \overline{v}\\
\implies&\forall m \in \sigma': Sender(m) = \overline{v}, ~ m \not\perp \overline{m} \\
\implies&\forall m \in \sigma \cup \{\overline{m}\}: Sender(m) = \overline{v}, ~ m \not\perp \overline{m} \\
\implies&\forall m \in \sigma: Sender(m) = \overline{v}, ~ m \not\perp \overline{m} \\
\implies&\forall m \in \sigma: Sender(m) = \overline{v}, \\
        &\neg (Sender(m) = Sender(\overline{m}) \land m \neq \overline{m} \land m \notin Justification(\overline{m}) \land \overline{m} \notin Justification(m)) \\
\implies&\forall m \in \sigma: Sender(m) = \overline{v}, \\
        &Sender(m) \neq Sender(\overline{m}) \lor m = \overline{m} \lor m \in Justification(\overline{m}) \lor \overline{m} \in Justification(m) \\
\implies&\forall m \in \sigma: Sender(m) = \overline{v}, \\
        &Sender(m) \neq Sender(\overline{m}) \lor m = \overline{m} \lor m \in Justification(\overline{m}) \lor \overline{m} \in Justification(m) \\
        &\land Sender(m) = \overline{v} \land Sender(\overline{m}) = \overline{v} \\
\implies&\forall m \in \sigma: Sender(m) = \overline{v}, \\
        &Sender(m) \neq Sender(\overline{m}) \lor m = \overline{m} \lor m \in Justification(\overline{m}) \lor \overline{m} \in Justification(m) \\
        &\land Sender(m) = Sender(\overline{m}) \\
\implies&\forall m \in \sigma: Sender(m) = \overline{v}, m = \overline{m} \lor m \in Justification(\overline{m}) \lor \overline{m} \in Justification(m) \\
\implies&\forall m \in \sigma: Sender(m) = \overline{v}, m = \overline{m} \lor m \in Justification(\overline{m}) \lor \overline{m} \in Justification(m) \\
        &\land m \in \sigma \land \overline{m} \notin \sigma \\
\implies&\forall m \in \sigma: Sender(m) = \overline{v}, m = \overline{m} \lor m \in Justification(\overline{m}) \lor \overline{m} \in Justification(m) \\
        &\land m \neq \overline{m} \\
\implies&\forall m \in \sigma: Sender(m) = \overline{v}, m \in Justification(\overline{m}) \lor \overline{m} \in Justification(m) \\
\implies&\forall m \in \sigma: Sender(m) = \overline{v}, m \in Justification(\overline{m}) \lor \overline{m} \in Justification(m) \\
        &\land \forall \sigma^* \in \Sigma, \forall m^* \in \sigma^*, Justification(m^*) \subseteq \sigma^* \\
\implies&\forall m \in \sigma: Sender(m) = \overline{v}, m \in Justification(\overline{m}) \lor \overline{m} \in Justification(m) \\
        &\land (\overline{m} \in Justification(m) \implies \overline{m} \in \sigma) \\
\implies&\forall m \in \sigma: Sender(m) = \overline{v}, m \in Justification(\overline{m}) \lor \overline{m} \in Justification(m) \\
        &\land (\overline{m} \notin \sigma \implies \overline{m} \notin Justification(m)) \\
\implies&\forall m \in \sigma: Sender(m) = \overline{v}, m \in Justification(\overline{m}) \lor \overline{m} \in Justification(m) \\
        &\land (\overline{m} \notin \sigma \implies \overline{m} \notin Justification(m)) \land \overline{m} \notin \sigma \\
\implies&\forall m \in \sigma: Sender(m) = \overline{v}, m \in Justification(\overline{m}) \lor \overline{m} \in Justification(m) \\
        &\land \overline{m} \notin Justification(m) \\
\implies&\forall m \in \sigma: Sender(m) = \overline{v}, m \in Justification(\overline{m})\\
\end{align}
\end{proof}
\fi


\begin{lemma}[$\overline{m}$ is $\overline{v}$'s latest message in $\sigma$']
$\forall (\sigma, \sigma') \in Minimal_t$, letting $\overline{m} =~!\sigma'\setminus\sigma$ and $\overline{v} = Sender(\overline{m})$,
$$
\overline{v} \notin E(\sigma') \implies \overline{m} = !L^H_M(\sigma')(\overline{v})))
$$
\end{lemma}

\begin{proof}
Note that $m \in Justification(m') \implies m' \notin Justification(m)$.
\begin{align}
        &\overline{v} \notin E(\sigma') \\
\implies&\overline{v} \notin E(\sigma') \land \sigma \in \Sigma ~\land \sigma \subseteq \sigma' \land \{\overline{m}\} \subseteq \sigma' \land \sigma \cap \{\overline{m}\} = \emptyset \\
\implies&\forall m_1 \in \sigma: Sender(m_1) = \overline{v}, \forall m_2 \in \{\overline{m}\}: Sender(m_2) = \overline{v}, m_1 \in Justification(m_2) \\
\implies&\forall m \in \sigma: Sender(m) = \overline{v}, m \in Justification(\overline{m}) \\
\implies&\forall m \in \sigma: Sender(m) = \overline{v}, \overline{m} \notin Justification(m) \\
\implies&\forall m \in \sigma: Sender(m) = \overline{v}, \neg Later(m, \overline{m}) \\
\implies&\forall m \in \sigma, \neg (Sender(m) = \overline{v} \land Later(m, \overline{m})) \\
\implies&\{m \in \sigma: Sender(m) = \overline{v} \land Later(m, \overline{m})\} = \emptyset \\
\implies&\{m \in \sigma: Sender(m) = \overline{v} \land Later(m, \overline{m})\} = \emptyset \land \forall m \in M, m \notin Justification(m) \\
\implies&\{m \in \sigma: Sender(m) = \overline{v} \land Later(m, \overline{m})\} = \emptyset \land \neg Later(\overline{m}, \overline{m}) \\
\implies&\{m \in \sigma: Sender(m) = \overline{v} \land Later(m, \overline{m})\} = \emptyset \land \{m \in \{\overline{m}\}: Sender(m) = \overline{v} \land Later(m, \overline{m})\} = \emptyset \\
\implies&\{m \in \sigma: Sender(m) = \overline{v} \land Later(m, \overline{m})\} \cup \{m \in \{\overline{m}\}: Sender(m) = \overline{v} \land Later(m, \overline{m})\} = \emptyset \\
\implies&\{m \in \sigma \cup \{\overline{m}\}: Sender(m) = \overline{v} \land Later(m, \overline{m})\} = \emptyset \\
\implies&\{m \in \sigma': Sender(m) = \overline{v} \land Later(m, \overline{m})\} = \emptyset \\
\implies&Later\_From(\overline{m}, \overline{v}, \sigma') = \emptyset \\
\end{align}
We will now be able to show that $\overline{m}$ is a latest honest message. Note that if a validator is not equivocating in $\sigma'$, the validators latest honest message is equal to the validators latest messages, by the defintion of latest honest message.
\begin{align}
        &\overline{v} \notin E(\sigma') \\
\implies&L^H_M(\sigma')(\overline{v}) = L_M(\sigma')(\overline{v}) \\
\implies&L^H_M(\sigma')(\overline{v}) = \{m \in \sigma' : Sender(m) = \overline{v} ~\land~ Later\_From(m,\overline{v},\sigma') = \emptyset\} \\
\implies&L^H_M(\sigma')(\overline{v}) = \{m \in \sigma \cup \{\overline{m}\} : Sender(m) = \overline{v} ~\land~ Later\_From(m,\overline{v},\sigma') = \emptyset\} \\
\implies&L^H_M(\sigma')(\overline{v}) = \{m \in \sigma: Sender(m) = \overline{v} ~\land~ Later\_From(m,\overline{v},\sigma') = \emptyset\} \\
        &\cup \{m \in \{\overline{m}\}: Sender(m) = \overline{v} ~\land~ Later\_From(m,\overline{v},\sigma') = \emptyset\} \\
\implies&L^H_M(\sigma')(\overline{v}) = \{m \in \sigma: Sender(m) = \overline{v} ~\land~ Later\_From(m,\overline{v},\sigma') = \emptyset\} \\
        &\cup \{m \in \{\overline{m}\}: Sender(m) = \overline{v} ~\land~ Later\_From(m,\overline{v},\sigma') = \emptyset\} \\
        &\land Sender(\overline{m}) = \overline{v} \land Later\_From(\overline{m},\overline{v},\sigma') = \emptyset \\
\implies& L^H_M(\sigma')(\overline{v}) = \{m \in \sigma: Sender(m) = \overline{v} ~\land~ Later\_From(m,\overline{v},\sigma') = \emptyset\} \cup \{\overline{m}\} \\
\implies& L^H_M(\sigma')(\overline{v}) = \{m \in \sigma: Sender(m) = \overline{v} ~\land~ Later\_From(m,\overline{v},\sigma') = \emptyset\} \cup \{\overline{m}\} \land |L^H_M(\sigma')(\overline{v})| = 1 \\
\implies& L^H_M(\sigma')(\overline{v}) = \{\overline{m}\} \\
\implies& \overline{m} = !L^H_M(\sigma')(\overline{v}) \\
\end{align}
\end{proof}


\begin{lemma}[Latest honest messages from non-equivocating messages are either the same as in their previous latest message, or later]
$\forall (\sigma, \sigma') \in Minimal_t, \forall v \in \mathcal{V}$, letting $\overline{m} =~!\sigma'\setminus\sigma$ and $\overline{v} = Sender(\overline{m})$

\begin{align}
\overline{v} \notin E(\sigma') \land v \notin E(\sigma) \implies&~!L^H_M(Justification(\overline{m}))(v) =~!L^H_M(!L^H_J(\sigma)(\overline{v}))(v) \\
&\lor Later(!L^H_M(Justification(\overline{m}))(v),~!L^H_M(!L^H_J(\sigma)(\overline{v}))(v))
\end{align}
\end{lemma}

\begin{proof}
First, we will show that $v \notin E(Justification(\overline{m}))$.
\begin{align}
          &v \notin E(\sigma) \\
  \implies&v \notin E(\sigma) \land Justification(\overline{m}) \subseteq \sigma \\
  \implies&v \notin E(\sigma) \land E(Justification(\overline{m})) \subseteq E(\sigma) \\
  \implies&v \notin E(Justification(\overline{m})) \\
\end{align}

Now, we will show that the justification $\overline{v}$'s previous latest message is a subset of the justification of this current message.
\begin{align}
          &\overline{v} \notin E(\sigma') \\
  \implies& \overline{v} \notin E(\sigma') \land \sigma \in \Sigma ~\land \sigma \subseteq \sigma' \land \{\overline{m}\} \subseteq \sigma' \land \sigma \cap \{\overline{m}\} = \emptyset \\
  \implies& \forall m_1 \in \sigma: Sender(m_1) = \overline{v}, \forall m_2 \in \{\overline{m}\}: Sender(m_2) = \overline{v}, m_1 \in Justification(m_2) \\
  \implies& \forall m \in \sigma: Sender(m) = \overline{v}, m \in Justification(\overline{m}) \\
  \implies& \forall m \in \sigma: Sender(m) = \overline{v}, m \in Justification(\overline{m}) \\
          &\land \forall m_1 \in M, \forall m_2 \in Justification(m_1), Justification(m_2) \subseteq Justification(m_1) \\
  \implies& \forall m \in \sigma: Sender(m) = \overline{v}, m \in Justification(\overline{m}) \land Justification(m) \subseteq Justification(\overline{m}) \\
  \implies& \forall m \in \sigma: Sender(m) = \overline{v}, Justification(m) \subseteq Justification(\overline{m}) \\
  \implies& \forall m \in \sigma: Sender(m) = \overline{v}, Justification(m) \subseteq Justification(\overline{m}) \\
          &\land !L^H_M(\sigma)(\overline{v}) \in \sigma \land Sender(!L^H_M(\sigma)(\overline{v})) = \overline{v} \\
  \implies& Justification(!L^H_M(\sigma)(\overline{v})) \subseteq Justification(\overline{m}) \\
  \implies& !L^H_J(\sigma)(\overline{v}) \subseteq Justification(\overline{m}) \\
\end{align}
Note the important implication of above:
$$
Justification(\overline{m}) = !L^H_J(\sigma)(\overline{v}) \cup (Justification(\overline{m}) \setminus !L^H_J(\sigma)(\overline{v}))
$$

We now consider the set:
$$
New = \{ m \in Justification(\overline{m}) \setminus !L^H_J(\sigma)(\overline{v}): Sender(m) = v \}
$$
That is, the messages validator $\overline{v}$ has newly seen from validator $v$. There are two cases to consider: when $New = \emptyset$, or when $New \neq \emptyset$, where each one of which correponds to one piece of the disjunction of the lemma.

\textbf{Case 1:} $New = \emptyset$.
\begin{align}
          &v \notin E(Justification(\overline{m})) \\
  \implies&L^H_M(Justification(\overline{m}))(v) = L_M(Justification(\overline{m}))(v) \\
  \implies&L^H_M(Justification(\overline{m}))(v) = \\
          &\{m \in Justification(\overline{m}) : Sender(m) = v \land Later\_From(m,v,Justification(\overline{m})) = \emptyset\} \\
  \implies&L^H_M(Justification(\overline{m}))(v) = \\
          &\{m \in Justification(\overline{m}) : Sender(m) = v \land Later\_From(m,v,Justification(\overline{m})) = \emptyset\} ~\land \\
          &Justification(\overline{m}) = !L^H_J(\sigma)(\overline{v}) \cup (Justification(\overline{m}) \setminus !L^H_J(\sigma)(\overline{v})) \\
  \implies&L^H_M(Justification(\overline{m}))(v) = \\
          &\{m \in !L^H_J(\sigma)(\overline{v}) \cup (Justification(\overline{m}) \setminus !L^H_J(\sigma)(\overline{v})) : Sender(m) = v \\
          &\land~ Later\_From(m,v,Justification(\overline{m})) = \emptyset\} \\
  \implies&L^H_M(Justification(\overline{m}))(v) = \\
          &\{m \in !L^H_J(\sigma)(\overline{v}) : Sender(m) = v \land Later\_From(m,v,Justification(\overline{m})) = \emptyset\} ~\cup \\
          &\{m \in Justification(\overline{m}) \setminus !L^H_J(\sigma)(\overline{v}) : Sender(m) = v \land Later\_From(m,v,Justification(\overline{m})) = \emptyset\}\\
\end{align}

We now show one of these terms is equal to the empty set.
\begin{align}
          &New = \emptyset \\
  \implies&\{m \in Justification(\overline{m}) \setminus !L^H_J(\sigma)(\overline{v}) : Sender(m) = v \} = \emptyset \\
  \implies&\{m \in Justification(\overline{m}) \setminus !L^H_J(\sigma)(\overline{v}) : Sender(m) = v \land Later\_From(m,v,Justification(\overline{m})) = \emptyset\} \\
          &= \emptyset \\
\end{align}

Thus, we can simplify:
\begin{align}
          &L^H_M(Justification(\overline{m}))(v) = \\
          &\{m \in !L^H_J(\sigma)(\overline{v}) : Sender(m) = v \land Later\_From(m,v,Justification(\overline{m})) = \emptyset\} ~\cup \\
          &\{m \in Justification(\overline{m}) \setminus !L^H_J(\sigma)(\overline{v}) : Sender(m) = v \land Later\_From(m,v,Justification(\overline{m})) = \emptyset\}\\
  \implies&L^H_M(Justification(\overline{m}))(v) = \\
          &\{m \in !L^H_J(\sigma)(\overline{v}) : Sender(m) = v \land Later\_From(m,v,Justification(\overline{m})) = \emptyset\} ~\cup \\
          &\{m \in Justification(\overline{m}) \setminus !L^H_J(\sigma)(\overline{v}) : Sender(m) = v \land Later\_From(m,v,Justification(\overline{m})) = \emptyset\}\\
          &\land \{m \in Justification(\overline{m}) \setminus !L^H_J(\sigma)(\overline{v}) : Sender(m) = v \land Later\_From(m,v,Justification(\overline{m})) = \emptyset\} \\
          &= \emptyset \\
  \implies&L^H_M(Justification(\overline{m}))(v) = \\
          &\{m \in !L^H_J(\sigma)(\overline{v}) : Sender(m) = v \land Later\_From(m,v,Justification(\overline{m})) = \emptyset\} \cup \emptyset \\
  \implies&L^H_M(Justification(\overline{m}))(v) = \\
          &\{m \in !L^H_J(\sigma)(\overline{v}) : Sender(m) = v \land Later\_From(m,v,Justification(\overline{m})) = \emptyset\} \\
\end{align}

We now show that, for any $m$:
$Later\_From(m,v,Justification(\overline{m})) = Later\_From(m,v,!L^H_J(\sigma)(\overline{v}))$

$\forall m \in M$,
\begin{align}
          &Later\_From(m,v,Justification(\overline{m})) = \\
          &\{m' \in Justification(\overline{m}) : Sender(m') = v \land Later(m', m)\} \\
  \implies&Later\_From(m,v,Justification(\overline{m})) = \\
          &\{m' \in Justification(\overline{m}) : Sender(m') = v \land Later(m', m)\} \\
          &\land Justification(\overline{m}) = !L^H_J(\sigma)(\overline{v}) \cup (Justification(\overline{m}) \setminus !L^H_J(\sigma)(\overline{v})) \\
  \implies&Later\_From(m,v,Justification(\overline{m})) = \\
          &\{m' \in !L^H_J(\sigma)(\overline{v}) \cup (Justification(\overline{m}) \setminus !L^H_J(\sigma)(\overline{v})) : Sender(m') = v \land Later(m', m)\} \\
  \implies&Later\_From(m,v,Justification(\overline{m})) = \\
          &\{m' \in !L^H_J(\sigma)(\overline{v}) : Sender(m') = v \land Later(m', m)\} \cup \\
          &\{m' \in Justification(\overline{m}) \setminus !L^H_J(\sigma)(\overline{v}): Sender(m') = v \land Later(m', m)\} \\
  \implies&Later\_From(m,v,Justification(\overline{m})) = \\
          &\{m' \in !L^H_J(\sigma)(\overline{v}) : Sender(m') = v \land Later(m', m)\} \cup \\
          &\{m' \in Justification(\overline{m}) \setminus !L^H_J(\sigma)(\overline{v}): Sender(m') = v \land Later(m', m)\} \\
          &\land \{ m \in Justification(\overline{m}) \setminus !L^H_J(\sigma)(\overline{v}): Sender(m) = v \} = \emptyset \\
  \implies&Later\_From(m,v,Justification(\overline{m})) = \\
          &\{m' \in !L^H_J(\sigma)(\overline{v}) : Sender(m') = v \land Later(m', m)\} \cup \\
          &\{m' \in Justification(\overline{m}) \setminus !L^H_J(\sigma)(\overline{v}): Sender(m') = v \land Later(m', m)\} \\
          &\land \{m' \in Justification(\overline{m}) \setminus !L^H_J(\sigma)(\overline{v}): Sender(m') = v \land Later(m', m)\} = \emptyset \\
  \implies&Later\_From(m,v,Justification(\overline{m})) = \\
          &\{m' \in !L^H_J(\sigma)(\overline{v}) : Sender(m') = v \land Later(m', m)\} \cup \emptyset \\
  \implies&Later\_From(m,v,Justification(\overline{m})) = \\
          &\{m' \in !L^H_J(\sigma)(\overline{v}) : Sender(m') = v \land Later(m', m)\} \\
  \implies&Later\_From(m,v,Justification(\overline{m})) = Later\_From(m,v,!L^H_J(\sigma)(\overline{v})) \\
\end{align}

This is enough for us to complete the proof
\begin{align}
          &L^H_M(Justification(\overline{m}))(v) = \\
          &\{m \in !L^H_J(\sigma)(\overline{v}) : Sender(m) = v \land Later\_From(m,v,Justification(\overline{m})) = \emptyset\}\\
  \implies&L^H_M(Justification(\overline{m}))(v) = \\
          &\{m \in !L^H_J(\sigma)(\overline{v}) : Sender(m) = v \land Later\_From(m,v,Justification(\overline{m})) = \emptyset\}\\
          &\land \forall m \in M, Later\_From(m,v,Justification(\overline{m})) = Later\_From(m,v,!L^H_J(\sigma)(\overline{v})) \\
  \implies&L^H_M(Justification(\overline{m}))(v) = \\
          &\{m \in !L^H_J(\sigma)(\overline{v}) : Sender(m) = v \land Later\_From(m,v,!L^H_J(\sigma)(\overline{v})) = \emptyset\}\\
  \implies&L^H_M(Justification(\overline{m}))(v) = L^H_M(!L^H_J(\sigma)(\overline{v}))(v) \\
  \implies&!L^H_M(Justification(\overline{m}))(v) = !L^H_M(!L^H_J(\sigma)(\overline{v}))(v) \\
\end{align}

\textbf{Case 2:} $New \neq \emptyset$
\begin{align}
          &v \notin E(Justification(\overline{m})) \\
  \implies&v \notin E(Justification(\overline{m})) ~\land~ !L^H_J(\sigma)(\overline{v}) \in \Sigma ~\land~ !L^H_J(\sigma)(\overline{v}) \subseteq Justification(\overline{m}) \\
          &\land (Justification(\overline{m}) \setminus !L^H_J(\sigma)(\overline{v})) \subseteq Justification(\overline{m}) \\
          &\land !L^H_J(\sigma)(\overline{v}) \cap (Justification(\overline{m}) \setminus !L^H_J(\sigma)(\overline{v})) = \emptyset \\
  \implies&\forall m_1 \in !L^H_J(\sigma)(\overline{v}) : Sender(m_1) = v, \forall m_2 \in Justification(\overline{m}) \setminus !L^H_J(\sigma)(\overline{v}) : Sender(m_2) = v, \\
          &m_1 \in Justification(m_2) \\
  \implies&\forall m_1 \in !L^H_J(\sigma)(\overline{v}) : Sender(m_1) = v, \forall m_2 \in \{m \in Justification(\overline{m}) \setminus !L^H_J(\sigma)(\overline{v}) : Sender(m_2) = v\}, \\
          &m_1 \in Justification(m_2) \\
  \implies&\forall m_1 \in !L^H_J(\sigma)(\overline{v}) : Sender(m_1) = v, \forall m_2 \in New, m_1 \in Justification(m_2) \\
  \implies&\forall m_1 \in !L^H_J(\sigma)(\overline{v}) : Sender(m_1) = v, \forall m_2 \in New, m_1 \in Justification(m_2) \\
          &\land !L^H_M(!L^H_J(\sigma)(\overline{v}))(v) \in !L^H_J(\sigma)(\overline{v}) \\
  \implies&\forall m_2 \in New, !L^H_M(!L^H_J(\sigma)(\overline{v}))(v) \in Justification(m_2) \\
  \implies&\forall m \in New, Later(m, !L^H_M(!L^H_J(\sigma)(\overline{v}))(v)) \\
\end{align}

It just remains to show that $!L^H_M(Justification(\overline{m}))(v) \in New$. We know it exists as ! is defined. Assume towards a contradiction that $!L^H_M(Justification(\overline{m}))(v) \notin New$.
\begin{align}
          &!L^H_M(Justification(\overline{m}))(v) \notin New \\
  \implies&!L^H_M(Justification(\overline{m}))(v) \notin \{ m \in Justification(\overline{m}) \setminus !L^H_J(\sigma)(\overline{v}): Sender(m) = v \} \\
  \implies&!L^H_M(Justification(\overline{m}))(v) \notin \{ m \in Justification(\overline{m}) \setminus !L^H_J(\sigma)(\overline{v}): Sender(m) = v \} \\
          &\land !L^H_M(Justification(\overline{m}))(v) \in Justification(\overline{m}) \\
  \implies&!L^H_M(Justification(\overline{m}))(v) \in \{ m \in !L^H_J(\sigma)(\overline{v}): Sender(m) = v \} \\
  \implies&!L^H_M(Justification(\overline{m}))(v) \in \{ m \in !L^H_J(\sigma)(\overline{v}): Sender(m) = v \} \\
          &\land \forall m_1 \in !L^H_J(\sigma)(\overline{v}) : Sender(m_1) = v, \forall m_2 \in New, m_1 \in Justification(m_2) \\
  \implies&!L^H_M(Justification(\overline{m}))(v) \in \{ m \in !L^H_J(\sigma)(\overline{v}): Sender(m) = v \} \\
          &\land \forall m_1 \in !L^H_J(\sigma)(\overline{v}) : Sender(m_1) = v, \forall m_2 \in New, m_1 \in Justification(m_2) \\
          &\land New \neq \emptyset \\
  \implies&!L^H_M(Justification(\overline{m}))(v) \in \{ m \in !L^H_J(\sigma)(\overline{v}): Sender(m) = v \} \\
          &\land \forall m_1 \in !L^H_J(\sigma)(\overline{v}) : Sender(m_1) = v, \exists m_2 \in New, m_1 \in Justification(m_2) \\
  \implies&\exists m \in New, !L^H_M(Justification(\overline{m}))(v) \in Justification(m) \\
  \implies&\exists m \in New, Later(m, !L^H_M(Justification(\overline{m}))(v)) \\
  \implies&\exists m \in New, Later(m, !L^H_M(Justification(\overline{m}))(v)) \land Sender(m) = v \\
  \implies&\{m \in New : Sender(m) = v \land Later(m, !L^H_M(Justification(\overline{m}))(v))\} \neq \emptyset \\
  \implies&\{m \in New : Sender(m) = v \land Later(m, !L^H_M(Justification(\overline{m}))(v))\} \neq \emptyset \\
          &\land New \subseteq Justification(\overline{m}) \\
  \implies&\{m \in Justification(\overline{m}) : Sender(m) = v \land Later(m, !L^H_M(Justification(\overline{m}))(v))\} \neq \emptyset \\
  \implies&Later\_From(!L^H_M(Justification(\overline{m}))(v), v, Justification(\overline{m})) \neq \emptyset
\end{align}
This is clearly a contradiction, as the latest message must have no later messages. Thus, we know that $!L^H_M(Justification(\overline{m}))(v) \in New$, and now we can complete the second case.

\begin{align}
          &\forall m \in New, Later(m, !L^H_M(!L^H_J(\sigma)(\overline{v}))(v)) \\
          &\land ~!L^H_M(Justification(\overline{m}))(v) \in New \\
  \implies& Later(!L^H_M(Justification(\overline{m}))(v), !L^H_M(!L^H_J(\sigma)(\overline{v}))(v))\\
\end{align}

Thus, we have completed the two cases:

\begin{align}
        &\overline{v} \notin E(\sigma') \land v \notin E(\sigma) \\
\implies& New = \emptyset \lor New \neq \emptyset \\
\implies& !L^H_M(Justification(\overline{m}))(v) = !L^H_M(!L^H_J(\sigma)(\overline{v}))(v) \\
        &\lor Later(!L^H_M(Justification(\overline{m}))(v),~!L^H_M(!L^H_J(\sigma)(\overline{v}))(v))
\end{align}

\end{proof}


\begin{lemma}
$\forall \sigma^* \in \Sigma, \forall m_1 \in \sigma^*, \forall m_2 \in \sigma^*$,
$$
Later(m_1, m_2) \implies m_1 \in Later\_From(m_2, Sender(m_1), \sigma^*)
$$
\end{lemma}

\begin{proof}
\begin{align}
          &Later(m_1, m_2) \\
  \implies& Sender(m_1) = Sender(m_1) \land Later(m_1, m_2) \\
  \implies& m_1 \in \sigma^* \land Sender(m_1) = Sender(m_1) \land Later(m_1, m_2) \\
  \implies& m_1 \in \{m' \in \sigma^* : Sender(m') = Sender(m_1) \land Later(m', m_2)\} \\
  \implies& m_1 \in Later\_From(m_2, Sender(m_1), \sigma^*) \\
\end{align}
\end{proof}


\begin{lemma}
$\forall \sigma \in \Sigma, \forall v \in \mathcal{V}$ such that ! is defined,
$$
p(Estimate(!L^H_M(\sigma)(v))) \implies v \in Agreeing(p, \sigma)
$$
\end{lemma}

\begin{proof}
\begin{align}
        &p(Estimate(!L^H_M(\sigma)(v))) \\
\implies& v \in \{ v \in \mathcal{V} : p(Estimate(!L^H_M(\sigma)(v))) \} \\
\implies& v \in \{ v \in \mathcal{V} : \forall m \in L^H_M(\sigma)(v), p(Estimate(m)) \} \\
\implies& v \in \{ v \in \mathcal{V} : \forall c \in L^H_E(\sigma)(v), p(c) \} \\
\implies& v \in Agreeing(p, \sigma) \\
\end{align}
\end{proof}


\begin{lemma}[Non-equivocating messages from clique members see clique agree]
$\forall p \in P_{\mathcal{C}} : Majority\_Driven^H(p), \forall (\sigma, \sigma') \in Minimal_t, \forall V \subseteq \mathcal{V}$, letting $\overline{m} =~!\sigma'\setminus\sigma$ and $\overline{v} = Sender(\overline{m})$
$$
Clique(1, V, p, \sigma) \land \overline{v} \in V \land \overline{v} \notin E(\sigma') \implies V \subseteq Agreeing(p, Justification(\overline{m}))
$$
\end{lemma}

%Clique\_Oracle_t(V, p, \sigma) &:\Leftrightarrow Clique(V, p, \sigma) \land Weight(V) >  (t + Weight(V))/2 - Weight(E(\sigma))

\begin{proof}
Our lemma gave us two cases, one where $\overline{v}$ updated $v$'s latest message in $Justification(\overline{m})$ from $!L^H_J(\sigma)(\overline{v})$, and one where it remained unchanged.

$\forall v \in V,$
\begin{align}
\overline{v} \in V \land \overline{v} \notin E(\sigma') &\implies~!L^H_M(Justification(\overline{m}))(v) =~!L^H_M(!L^H_J(\sigma)(\overline{v}))(v) \\
&~~\lor Later(!L^H_M(Justification(\overline{m}))(v),~!L^H_M(!L^H_J(\sigma)(\overline{v})(v))
\end{align}

We will use the ``mutually-observed agreement" part of the Clique definition to show that the unchanged messages agree, and the ``no later disagreeing'' part to show that updated messages also agree.

For the first case where the latest messages from validator $v$ are the same:
\begin{align}
        &Clique(1, V, p, \sigma) \land \overline{v} \in V \land \overline{v} \notin E(\sigma') \\
        &\land !L^H_M(Justification(\overline{m}))(v) =~!L^H_M(!L^H_J(\sigma)(\overline{v}))(v) \\
\implies&!L^H_M(Justification(\overline{m}))(v) =~!L^H_M(!L^H_J(\sigma)(\overline{v}))(v) \\
        &\land \forall v' \in V, V \subseteq Agreeing(p, !L^H_J(\sigma)(v')) \\
\implies&!L^H_M(Justification(\overline{m}))(v) =~!L^H_M(!L^H_J(\sigma)(\overline{v}))(v) \\
        &\land \forall v' \in V, v \in Agreeing(p, !L^H_J(\sigma)(v')) \\
\implies&!L^H_M(Justification(\overline{m}))(v) =~!L^H_M(!L^H_J(\sigma)(\overline{v}))(v) \\
        &\land v \in Agreeing(p, !L^H_J(\sigma)(\overline{v})) \\
\implies&!L^H_M(Justification(\overline{m}))(v) =~!L^H_M(!L^H_J(\sigma)(\overline{v}))(v) \\
        &\land v \in \{v' \in \mathcal{V} : \forall c \in L^H_E(!L^H_J(\sigma)(\overline{v}))(v'), p(c)\}  \\
\implies&!L^H_M(Justification(\overline{m}))(v) =~!L^H_M(!L^H_J(\sigma)(\overline{v}))(v) \\
        &\land \forall c \in L^H_E(!L^H_J(\sigma)(\overline{v}))(v), p(c)  \\
\implies&!L^H_E(Justification(\overline{m}))(v) =~!L^H_E(!L^H_J(\sigma)(\overline{v}))(v) \\
        &\land \forall c \in L^H_E(!L^H_J(\sigma)(\overline{v}))(v), p(c)  \\
\implies&\forall c \in L^H_E(Justification(\overline{m}))(v), p(c)  \\
\implies&v \in \{v' \in \mathcal{V} : \forall c \in L^H_E(Justification(\overline{m}))(v'), p(c)\}  \\
\implies&v \in Agreeing(p, Justification(\overline{m}))
\end{align}

And in the case when there is a new later message from $v$:
\begin{align}
        &Clique(1, V, p, \sigma) \land \overline{v} \in V \land \overline{v} \notin E(\sigma') \\
        &\land Later(~!L^H_M(Justification(\overline{m}))(v),~!L^H_M(!L^H_J(\sigma)(\overline{v}))(v)) \\
\implies& Later(~!L^H_M(Justification(\overline{m}))(v),~!L^H_M(!L^H_J(\sigma)(\overline{v}))(v)) \\
        &\land \forall v' \in V, \forall v'' \in V,Later\_Disagreeing(p,~!L^H_M(!L^H_J(\sigma)(v'))(v''), v'', \sigma) = \emptyset \\
\implies&Later(!L^H_M(Justification(\overline{m}))(v),~!L^H_M(!L^H_J(\sigma)(\overline{v})(v))) \\
        &\land Later\_Disagreeing(p,~!L^H_M(!L^H_J(\sigma)(\overline{v}))(v), v, \sigma) = \emptyset \\
\implies& Later(!L^H_M(Justification(\overline{m}))(v),~!L^H_M(!L^H_J(\sigma)(\overline{v})(v))) \\
        &\land \{m' \in Later\_From(!L^H_M(!L^H_J(\sigma)(\overline{v}))(v), v, \sigma) : \neg p(Estimate(m')) \}  = \emptyset \\
\implies& Later(!L^H_M(Justification(\overline{m}))(v),~!L^H_M(!L^H_J(\sigma)(\overline{v})(v))) \\
        &\land \nexists m' \in Later\_From(!L^H_M(!L^H_J(\sigma)(\overline{v}))(v), v, \sigma), \neg p(Estimate(m')) \\
\implies& Later(!L^H_M(Justification(\overline{m}))(v),~!L^H_M(!L^H_J(\sigma)(\overline{v})(v))) \\
        &\land \forall m' \in Later\_From(!L^H_M(!L^H_J(\sigma)(\overline{v}))(v), v, \sigma), p(Estimate(m')) \\
\implies& Later(!L^H_M(Justification(\overline{m}))(v),~!L^H_M(!L^H_J(\sigma)(\overline{v})(v))) \\
        &\land \forall m' \in Later\_From(!L^H_M(!L^H_J(\sigma)(\overline{v}))(v), v, \sigma), p(Estimate(m')) \\
        &\land \forall \sigma^* \in \Sigma, \forall m_1 \in \sigma^*, \forall m_2 \in \sigma^*, Later(m_1, m_2) \implies m_1 \in Later\_From(m_2, Sender(m_1), \sigma^*) \\
\implies&!L^H_M(Justification(\overline{m}))(v) \in Later\_From(!L^H_M(!L^H_J(\sigma)(\overline{v}))(v), v, \sigma) \\
        &\land \forall m' \in Later\_From(!L^H_M(!L^H_J(\sigma)(\overline{v}))(v), v, \sigma), p(Estimate(m')) \\
\implies&p(Estimate(!L^H_M(Justification(\overline{m}))(v))) \\
\implies&p(Estimate(!L^H_M(Justification(\overline{m}))(v))) \land \\
        &(\forall \sigma^* \in \Sigma, \forall v' \in \mathcal{V}, p(Estimate(!L^H_M(\sigma^*)(v'))) \implies v' \in Agreeing(p, \sigma^*)) \\
\implies&v \in Agreeing(p, Justification(\overline{m}))
\end{align}


Since in both of our cases we have $v \in Agreeing(p, Justification(m))$, we the following result

\begin{align}
        &Clique(1, V, p, \sigma) \land \overline{v} \in V \land \overline{v} \notin E(\sigma') \\
\implies&!L^H_M(Justification(\overline{m}))(v) =~!L^H_M(!L^H_J(\sigma)(\overline{v}))(v) \\
        &\lor Later(!L^H_M(Justification(\overline{m}))(v),~!L^H_M(!L^H_J(\sigma)(\overline{v})(v))\\
\implies&\forall v \in V, v \in Agreeing(p, Justification(\overline{m})) \\
        &\lor \forall v \in V, v \in Agreeing(p, Justification(\overline{m})) \\
\implies&\forall v \in V, v \in Agreeing(p, Justification(\overline{m}))
\end{align}

Which gives us our result:

$$
Clique(1, V, p, \sigma) \land \overline{v} \in V \land \overline{v} \notin E(\sigma') \implies V \subseteq Agreeing(p, Justification(\overline{m}))
$$

\end{proof}


%---------------------------------------------------------------------------------------------------------------------
%---------------------------------------------------------------------------------------------------------------------
%---------------------------------------------------------------------------------------------------------------------

\subsubsection{Non-equivocating messages from majority clique members agree}


\begin{lemma}[New messages from majority clique members agree]
$\forall p \in P_{\mathcal{C}} : Majority\_Driven^H(p), \forall (\sigma, \sigma') \in Minimal_t, \forall V \subseteq \mathcal{V}$, letting $\overline{m} =~!\sigma'\setminus\sigma$ and $\overline{v} = Sender(\overline{m})$.
\begin{align}
Clique(1, V, p, \sigma) \land \overline{v} \in V \land \overline{v} \notin E(\sigma')& \\
\land \forall v \in V, Majority^H(V, !L^H_J(\sigma)(v)) & \implies \overline{v} \in Agreeing(p, \sigma')
\end{align}

\end{lemma}

\begin{proof}

We note that $\overline{m}$'s justification includes at least the equivocating weight as $\overline{v}$'s previous latest message.
\begin{align}
        &\overline{v} \notin E(\sigma') \\
\implies&Justification(\overline{m}) \supseteq !L^H_J(\sigma)(\overline{v}) \\
\implies&E(Justification(\overline{m})) \supseteq E(!L^H_J(\sigma)(\overline{v})) \\
\implies&Weight(E(Justification(\overline{m}))) \geq Weight(E(!L^H_J(\sigma)(\overline{v})))
\end{align}

An we use this to show that $V$ has a majority of the non-equivocating weight, in $\overline{m}$'s justification.

\begin{align}
          &\overline{v} \in V \land \forall v \in V, Majority^H(V, !L^H_J(\sigma)(v)) \\
\implies&Majority^H(V, !L^H_J(\sigma)(\overline{v})) \\
\implies&Weight(V) > Weight(\mathcal{V} \setminus E(!L^H_J(\sigma)(\overline{v})))/2 \\
\implies&Weight(V) > Weight(\mathcal{V})/2 - Weight(E(!L^H_J(\sigma)(\overline{v})))/2 \\
\implies&Weight(V) > Weight(\mathcal{V})/2 - Weight(E(!L^H_J(\sigma)(\overline{v})))/2 \\
        &\land Weight(E(Justification(\overline{m}))) \geq Weight(E(!L^H_J(\sigma)(\overline{v}))) \\
\implies&Weight(V) > Weight(\mathcal{V})/2 - Weight(E(Justification(\overline{m})))/2 \\
\implies&Weight(V) > Weight(\mathcal{V} \setminus E(Justification(\overline{m})))/2 \\
\implies&Weight(V) > Weight(\mathcal{V} \setminus E(Justification(\overline{m})))/2 \\
\implies&Majority^H(V, Justification(\overline{m}))
\end{align}

We will now use our previous lemma, this result, and the definition of majority-driven properties to complete the proof:

\begin{align}
        &Clique(1, V, p, \sigma) \land \overline{v} \in V \land \overline{v} \notin E(\sigma') \\
\implies&V \subseteq Agreeing(p, Justification(\overline{m}))\\
\implies&Weight(V) \leq Weight(Agreeing(p, Justification(\overline{m})))\\
\implies&Weight(V) \leq Weight(Agreeing(p, Justification(\overline{m})))\\
        &\land Majority^H(V, Justification(\overline{m})) \\
\implies&Weight(V) \leq Weight(Agreeing(p, Justification(\overline{m})))\\
        &\land Weight(V) > Weight(\mathcal{V} \setminus E(Justification(\overline{m})))/2 \\
\implies&Weight(Agreeing(p, Justification(\overline{m}))) > Weight(\mathcal{V} \setminus E(Justification(\overline{m})))/2 \\
\implies&Majority^H(Agreeing(p, Justification(\overline{m})), Justification(\overline{m})) \\
\implies&Majority^H(Agreeing(p, Justification(\overline{m})), Justification(\overline{m})) \\
        &\land [\forall \sigma^* \in \Sigma, Majority^H(Agreeing(p, \sigma^*), \sigma^*) \implies \forall c \in \mathcal{E}(\sigma^*), p(c)] \\
\implies&\forall c \in \mathcal{E}(Justification(\overline{m})), p(c) \\
\implies&\forall c \in \mathcal{E}(Justification(\overline{m})), p(c) \land Estimate(\overline{m}) \in \mathcal{E}(Justification(\overline{m})) \\
\implies&p(Estimate(\overline{m})) \\
\implies&p(Estimate(!L^H_M(\sigma')(\overline{v}))) \\
\implies&p(Estimate(!L^H_M(\sigma')(\overline{v}))) \land \\
        &(\forall \sigma^* \in \Sigma, \forall v' \in \mathcal{V}, p(Estimate(!L^H_M(\sigma^*)(v'))) \implies v' \in Agreeing(p, \sigma^*)) \\
\implies&\overline{v} \in Agreeing(p, \sigma') \\
\end{align}
\end{proof}



%---------------------------------------------------------------------------------------------------------------------
%---------------------------------------------------------------------------------------------------------------------
%---------------------------------------------------------------------------------------------------------------------
%---------------------------------------------------------------------------------------------------------------------
%---------------------------------------------------------------------------------------------------------------------
%---------------------------------------------------------------------------------------------------------------------

\subsubsection{Honest messages from majority clique members do not break the clique}


\begin{lemma}
$$
\overline{v} \notin E(\sigma') \implies !L^H_M(Justification(\overline{m}))(\overline{v}) = !L^H_M(\sigma)(\overline{v})
$$
\end{lemma}

\begin{proof}
First, we show that $\overline{v}$ is not equivocating in $Justification(\overline{m})$.
\begin{align}
          &\overline{v} \notin E(\sigma') \\
  \implies&\overline{v} \notin E(\sigma') \land Justification(\overline{m}) \subseteq \sigma \land \sigma \subset \sigma' \\
  \implies&\overline{v} \notin E(\sigma') \land Justification(\overline{m}) \subset \sigma' \\
  \implies&\overline{v} \notin E(\sigma') \land E(Justification(\overline{m})) \subseteq E(\sigma') \\
  \implies&\overline{v} \notin E(Justification(\overline{m})) \\
\end{align}

Now, assume towards a contradiction:
$$
\exists m \in Justification(\overline{m}): Sender(m) = \overline{v} \land Later(m, !L^H_M(\sigma)(\overline{v}))
$$
\begin{align}
  &\exists m \in Justification(\overline{m}): Sender(m) = \overline{v} \land Later(m, !L^H_M(\sigma)(\overline{v})) \\
  \implies& \exists m \in Justification(\overline{m}): Sender(m) = \overline{v} \land Later(m, !L^H_M(\sigma)(\overline{v})) \land Justification(\overline{m}) \subseteq \sigma \\
  \implies& \exists m \in \sigma: Sender(m) = \overline{v} \land Later(m, !L^H_M(\sigma)(\overline{v})) \\
  \implies& \{m \in \sigma : Sender(m) = \overline{v} \land Later(m, !L^H_M(\sigma)(\overline{v}))\} \neq \emptyset \\
  \implies& Later\_From(!L^H_M(\sigma)(\overline{v})), \overline{v}, \sigma) \neq \emptyset \\
\end{align}
Thus, this is clearly a contradiction, as the latest message in a state cannot have any later messages. Thus, we know that:
$$
\neg (\exists m \in Justification(\overline{m}): Sender(m) = \overline{v} \land Later(m, !L^H_M(\sigma)(\overline{v})))
$$
which is enough to complete the proof.
\begin{align}
          &\neg (\exists m \in Justification(\overline{m}): Sender(m) = \overline{v} \land Later(m, !L^H_M(\sigma)(\overline{v}))) \\
  \implies&\nexists m \in Justification(\overline{m}): Sender(m) = \overline{v} \land Later(m, !L^H_M(\sigma)(\overline{v})) \\
  \implies&\{m \in Justification(\overline{m}) : Sender(m) = \overline{v} \land  Later(m, !L^H_M(\sigma)(\overline{v}))\} = \emptyset \\
  \implies&Later\_From(!L^H_M(\sigma)(\overline{v}), \overline{v}, Justification(\overline{m})) = \emptyset \\
  \implies&Sender(!L^H_M(\sigma)(\overline{v})) = \overline{v} \land Later\_From(!L^H_M(\sigma)(\overline{v}), \overline{v}, Justification(\overline{m})) = \emptyset \\
  \implies&!L^H_M(\sigma)(\overline{v}) \in \{m \in Justification(\overline{m}) : Sender(m) = \overline{v} \land Later\_From(m, \overline{v}, Justification(\overline{m})) = \emptyset\} \\
  \implies&!L^H_M(\sigma)(\overline{v}) \in L_M(Justification(\overline{m}))(\overline{v}) \\
  \implies&!L^H_M(\sigma)(\overline{v}) \in L_M(Justification(\overline{m}))(\overline{v}) \land \overline{v} \notin E(Justification(\overline{m})) \\
  \implies&!L^H_M(\sigma)(\overline{v}) \in L_M(Justification(\overline{m}))(\overline{v}) \land L^H_M(Justification(\overline{m}))(\overline{v}) = L_M(Justification(\overline{m}))(\overline{v}) \\
  \implies&!L^H_M(\sigma)(\overline{v}) \in L^H_M(Justification(\overline{m}))(\overline{v})\\
  \implies&!L^H_M(\sigma)(\overline{v}) \in L^H_M(Justification(\overline{m}))(\overline{v}) \land |L^H_M(Justification(\overline{m}))(\overline{v})| = 1\\
  \implies&!L^H_M(Justification(\overline{m}))(\overline{v}) = !L^H_M(\sigma)(\overline{v})\\
\end{align}
\end{proof}


\begin{lemma}
$$
\overline{v} \notin E(\sigma') \implies Later(\overline{m}, !L^H_M(\sigma)(\overline{v}))
$$
\end{lemma}

\begin{proof}
\begin{align}
          &\overline{v} \notin E(\sigma') \\
  \implies&\overline{v} \notin E(\sigma') \land \sigma \subseteq \sigma' \land \sigma \in \Sigma \land \{\overline{m}\} \subseteq \sigma' \land \sigma \cap \{\overline{m}\} = \emptyset \\
  \implies&\forall m_1 \in \sigma : Sender(m_1) = \overline{v}, \forall m_2 \in \{\overline{m}\} : Sender(m_2) = \overline{v}, m_1 \in Justification(m_2) \\
  \implies&\forall m \in \sigma : Sender(m) = \overline{v}, m \in Justification(\overline{m}) \\
  \implies&\forall m \in \sigma : Sender(m) = \overline{v}, m \in Justification(\overline{m}) \\
          &\land !L^H_M(\sigma)(\overline{v}) \in \sigma \land Sender(!L^H_M(\sigma)(\overline{v})) = \overline{v}\\
  \implies& !L^H_M(\sigma)(\overline{v}) \in Justification(\overline{m}) \\
  \implies& Later(\overline{m}, !L^H_M(\sigma)(\overline{v})) \\
\end{align}
\end{proof}

\begin{lemma}[Nothing later than latest honest message]
$\forall v \in \mathcal{V}, \forall \sigma \in \Sigma, \forall m \in M$
$$
  v \notin E(\sigma) \implies Later\_From(!L^H_M(\sigma)(v), v, \sigma) = \emptyset
$$
We note that this lemma is used in some important contradictions above, and should be moved up at some point :)
\end{lemma}

\begin{proof}
\begin{align}
          &v \notin E(\sigma) \\
  \implies&L^H_M(\sigma)(v) = L_M(\sigma)(v) \\
  \implies&L^H_M(\sigma)(v) = \{m \in \sigma : Sender(m) = v ~\land~ Later\_From(m,v,\sigma) = \emptyset\} \\
  \implies&!L^H_M(\sigma)(v) = !\{m \in \sigma : Sender(m) = v ~\land~ Later\_From(m,v,\sigma) = \emptyset\} \\
  \implies&Later\_From(!L^H_M(\sigma)(v),v,\sigma) = \emptyset
\end{align}
\end{proof}


\begin{lemma}[Later messages for sender is just the new message]
$$
\overline{v} \notin E(\sigma') \implies Later\_From(!L^H_M(\sigma)(\overline{v}), \overline{v}, \sigma') = \{\overline{m}\}
$$
\end{lemma}

\begin{proof}
\begin{align}
        &\overline{v} \notin E(\sigma') \\
\implies&Later\_From(!L^H_M(\sigma)(\overline{v}), \overline{v}, \sigma') = \{m \in \sigma' : Sender(m) = \overline{v} \land Later(m, !L^H_M(\sigma)(\overline{v}))\} \\
\implies&Later\_From(!L^H_M(\sigma)(\overline{v}), \overline{v}, \sigma') = \{m \in \sigma \cup \{\overline{m}\} : Sender(m) = \overline{v} \land Later(m, !L^H_M(\sigma)(\overline{v}))\} \\
\implies&Later\_From(!L^H_M(\sigma)(\overline{v}), \overline{v}, \sigma') = \{m \in \sigma : Sender(m) = \overline{v} \land Later(m, !L^H_M(\sigma)(\overline{v}))\} \\
        &\cup \{\overline{m} : Sender(\overline{m}) = \overline{v} \land Later(\overline{m}, !L^H_M(\sigma)(\overline{v}))\} \\
\implies&Later\_From(!L^H_M(\sigma)(\overline{v}), \overline{v}, \sigma') = \{m \in \sigma : Sender(m) = \overline{v} \land Later(m, !L^H_M(\sigma)(\overline{v}))\} \\
        &\cup \{\overline{m}\} \\
\implies&Later\_From(!L^H_M(\sigma)(\overline{v}), \overline{v}, \sigma') = Later\_From(!L^H_M(\sigma)(\overline{v}), \overline{v}, \sigma) \cup \{\overline{m}\} \\
\implies&Later\_From(!L^H_M(\sigma)(\overline{v}), \overline{v}, \sigma') = \emptyset \cup \{\overline{m}\} \\
\implies&Later\_From(!L^H_M(\sigma)(\overline{v}), \overline{v}, \sigma') = \{\overline{m}\} \\
\end{align}
\end{proof}

\begin{lemma}[Later disagreeing is monotonic]
$\forall v \in V, \forall \sigma \in \Sigma, \forall m_1 \in M$,
$$
  m_1 \in Justification(m_2) \implies Later\_Disagreeing(p,m_2,v,\sigma) \subseteq Later\_Disagreeing(p,m_1,v,\sigma)
$$
\end{lemma}

\begin{proof}
\begin{align}
        &m_1 \in Justification(m_2) \\
\implies&Later\_Disagreeing(p,m_2,v,\sigma) = \{m' \in Later\_From(m_2, v, \sigma) : \neg p(Estimate(m')) \} \\
\implies&Later\_Disagreeing(p,m_2,v,\sigma) = \{m' \in \{m'' \in \sigma : Sender(m'') = v \land Later(m'', m_2)\} : \neg p(Estimate(m')) \} \\
\implies&Later\_Disagreeing(p,m_2,v,\sigma) = \{m' \in \sigma : Sender(m') = v \land Later(m', m_2) \land \neg p(Estimate(m')) \} \\
\implies&Later\_Disagreeing(p,m_2,v,\sigma) = \{m' \in \sigma : Sender(m') = v \land \neg p(Estimate(m')) \} \cap \{m' \in \sigma : Later(m', m_2)\} \\
\implies&Later\_Disagreeing(p,m_2,v,\sigma) = \{m' \in \sigma : Sender(m') = v \land \neg p(Estimate(m')) \} \cap \{m' \in \sigma : Later(m', m_2)\} \\
        &\land \forall m \in M, m_1 \in Justification(m_2) \implies (Later(m, m_2) \implies Later(m, m_1)) \\
\implies&Later\_Disagreeing(p,m_2,v,\sigma) = \{m' \in \sigma : Sender(m') = v \land \neg p(Estimate(m')) \} \cap \{m' \in \sigma : Later(m', m_2)\} \\
        &\land \forall m \in M, Later(m, m_2) \implies Later(m, m_1) \\
\implies&Later\_Disagreeing(p,m_2,v,\sigma) = \{m' \in \sigma : Sender(m') = v \land \neg p(Estimate(m')) \} \cap \{m' \in \sigma : Later(m', m_2)\} \\
        &\land \{m' \in \sigma : Later(m', m_2)\} \subseteq \{m' \in \sigma : Later(m', m_1)\} \\
\implies&Later\_Disagreeing(p,m_2,v,\sigma) \subseteq \{m' \in \sigma : Sender(m') = v \land \neg p(Estimate(m')) \} \cap \{m' \in \sigma : Later(m', m_1)\} \\
\implies&Later\_Disagreeing(p,m_2,v,\sigma) \subseteq \{m' \in \sigma : Sender(m') = v \land Later(m', m_1) \land \neg p(Estimate(m')) \} \\
\implies&Later\_Disagreeing(p,m_2,v,\sigma) \subseteq Later\_Disagreeing(p,m_1,v,\sigma) \\
\end{align}
\end{proof}


\begin{lemma}
\begin{align}
          &\overline{v} \notin E(\sigma') \land v' \notin E(\sigma) \\
          &Later\_Disagreeing(p, L^H_M(L^H_J(\sigma)(\overline{v}))(v'), v', \sigma) = \emptyset \\
  \implies&Later\_Disagreeing(p,~!L^H_M(Justification(\overline{m}))(v'), v', \sigma') = \emptyset \\
\end{align}
\end{lemma}



\begin{proof}
Recall the previous result:
\begin{align}
          &\overline{v} \notin E(\sigma') \land v' \notin E(\sigma) \\
  \implies&!L^H_M(Justification(\overline{m}))(v') = !L^H_M(!L^H_J(\sigma)(\overline{v}))(v') \\
          &\lor Later(!L^H_M(Justification(\overline{m}))(v'),~!L^H_M(!L^H_J(\sigma)(\overline{v}))(v')) \\
\end{align}

We will handle each case:

\begin{align}
          &Later\_Disagreeing(p, L^H_M(L^H_J(\sigma)(\overline{v}))(v'), v', \sigma) = \emptyset \\
          &\land !L^H_M(Justification(\overline{m}))(v') = !L^H_M(!L^H_J(\sigma)(\overline{v}))(v') \\
  \implies&Later\_Disagreeing(p, !L^H_M(Justification(\overline{m}))(v'), v', \sigma) = \emptyset \\
\end{align}

In the second case:
\begin{align}
          &Later\_Disagreeing(p, L^H_M(L^H_J(\sigma)(\overline{v}))(v'), v', \sigma) = \emptyset \\
          &\land Later(!L^H_M(Justification(\overline{m}))(v'),~!L^H_M(!L^H_J(\sigma)(\overline{v}))(v')) \\
  \implies&Later\_Disagreeing(p, L^H_M(L^H_J(\sigma)(\overline{v}))(v'), v', \sigma) = \emptyset \\
          &\land !L^H_M(!L^H_J(\sigma)(\overline{v}))(v') \in Justification(!L^H_M(Justification(\overline{m}))(v')) \\
  \implies&Later\_Disagreeing(p, L^H_M(L^H_J(\sigma)(\overline{v}))(v'), v', \sigma) = \emptyset \\
          &Later\_Disagreeing(p,!L^H_M(Justification(\overline{m}))(v'),v',\sigma) \subseteq Later\_Disagreeing(p,!L^H_M(!L^H_J(\sigma)(\overline{v}))(v'),v',\sigma) \\
  \implies&Later\_Disagreeing(p,!L^H_M(Justification(\overline{m}))(v'),v',\sigma) \subseteq \emptyset \\
  \implies&Later\_Disagreeing(p,!L^H_M(Justification(\overline{m}))(v'),v',\sigma) = \emptyset \\
\end{align}

Thus concluding the proof.
\end{proof}


\begin{lemma}[New non-equivocating latest messages from members of majority clique don't break the clique]
$\forall p \in P_{\mathcal{C}} : Majority\_Driven^H(p), \forall \sigma \in \Sigma_t, \forall (\sigma, \sigma') \in Minimal_t,\forall V \subseteq \mathcal{V}$, letting $\overline{m} =~!\sigma'\setminus\sigma$, and $\overline{v} = Sender(\overline{m})$,

\begin{align}
Clique(1, V, p, \sigma) \land \overline{v} \in V \land \overline{v} \notin E(\sigma') &\\
\land ~\forall v \in V, Majority^H(V, !L^H_J(\sigma)(v)) & \implies Clique(1, V, p, \sigma')
\end{align}
\end{lemma}

\begin{proof}

From our previous result,

$$
Clique(1, V, p, \sigma) \implies Clique(1, V \setminus \{\overline{v}\}, p, \sigma')
$$


Thus, we know that
\begin{align}
        &Clique(1, V, p, \sigma) \\
\implies&Clique(1, V \setminus \{\overline{v}\}, p, \sigma') \\
\implies&\forall v \in V \setminus \{\overline{v}\}, V \setminus \{\overline{v}\} \subseteq Agreeing(p, !L^H_J(\sigma')(v)) \\
        &~~~~~\land \forall v' \in V \setminus \{\overline{v}\}, Later\_Disagreeing(p,~!L^H_M(!L^H_J(\sigma')(v))(v'), v', \sigma') = \emptyset \\
\end{align}

To complete the proof, we must show that $\overline{v}$ is a member of the clique. Thus, it only remains to show the following:

\textbf{1:} $\forall v \in \{\overline{v}\}, V \subseteq Agreeing(p, !L^H_J(\sigma')(v))$

\begin{align}
          &Clique(1, V, p, \sigma) \land \overline{v} \in V \land \overline{v} \notin E(\sigma') \\
  \implies&V \subseteq Agreeing(p, Justification(\overline{m})) \land \overline{m} = !L^H_M(\sigma')(\overline{v})))\\
  \implies&V \subseteq Agreeing(p, Justification(!L^H_M(\sigma')(\overline{v}))))) \\
  \iff& \forall v \in \{\overline{v}\}, V \subseteq Agreeing(p, !L^H_J(\sigma')(v)) \\
\end{align}

\textbf{2:} $\forall v \in V \setminus \{\overline{v}\}, \{\overline{v}\} \subseteq Agreeing(p, !L^H_J(\sigma')(v))$

\begin{align}
          &Clique(1, V, p, \sigma) \land \overline{v} \in V \land \overline{v} \notin E(\sigma') \\
  \implies&\forall v \in V, V \subseteq Agreeing(p, !L^H_J(\sigma)(v)) \\
          &\land V \setminus \{\overline{v}\} \subseteq V \land \{\overline{v}\} \subseteq V \\
  \implies&\forall v \in V \setminus \{\overline{v}\}, \{\overline{v}\} \subseteq Agreeing(p, !L^H_J(\sigma)(v)) \\
          &\land \forall v \in V \setminus \{\overline{v}\}, !L^H_J(\sigma)(v) = !L^H_J(\sigma')(v) \\
  \implies&\forall v \in V \setminus \{\overline{v}\}, \{\overline{v}\} \subseteq Agreeing(p, !L^H_J(\sigma')(v)) \\
\end{align}

\textbf{3:} $\forall v \in \{\overline{v}\}, \forall v' \in \{\overline{v}\}, Later\_Disagreeing(p,~!L^H_M(!L^H_J(\sigma')(v))(v'), v', \sigma') = \emptyset$


To prove this condition, first observe:
\begin{align}
&\forall v \in \{\overline{v}\}, \forall v' \in \{\overline{v}\}, Later\_Disagreeing(p,~!L^H_M(!L^H_J(\sigma')(v))(v'), v', \sigma') = \emptyset \\
&\iff Later\_Disagreeing(p,~!L^H_M(!L^H_J(\sigma')(\overline{v}))(\overline{v}), \overline{v}, \sigma') = \emptyset \\
\end{align}

And then:
\begin{align}
 &Later\_Disagreeing(p,~!L^H_M(!L^H_J(\sigma')(\overline{v}))(\overline{v}), \overline{v}, \sigma') \\
=&Later\_Disagreeing(p,~!L^H_M(Justification(\overline{m}))(\overline{v}), \overline{v}, \sigma') \\
=& Later\_Disagreeing(p,~!L^H_M(\sigma)(\overline{v}), \overline{v}, \sigma') \\
=& \{ m \in Later\_From(!L^H_M(\sigma)(\overline{v}), \overline{v}, \sigma') : \neg p(Estimate(m)) \}\\
=& \{ m \in \{\overline{m}\} : \neg p(Estimate(m)) \}\\
=& \{ m \in \{\overline{m}\} : \neg p(Estimate(m)) \land p(Estimate(\overline{m})) \} \\
=& \emptyset
\end{align}

\textbf{4:} $\forall v \in \{\overline{v}\}, \forall v' \in V \setminus \{\overline{v}\}, Later\_Disagreeing(p,~!L^H_M(!L^H_J(\sigma')(v))(v'), v', \sigma') = \emptyset$

Again, first observe
\begin{align}
&\forall v \in \{\overline{v}\}, \forall v' \in V \setminus \{\overline{v}\}, Later\_Disagreeing(p,~!L^H_M(!L^H_J(\sigma')(v))(v'), v', \sigma') = \emptyset \\
&\iff \forall v' \in V \setminus \{\overline{v}\}, Later\_Disagreeing(p,~!L^H_M(!L^H_J(\sigma')(\overline{v}))(v'), v', \sigma') = \emptyset \\
\end{align}

$\forall v' \in V \setminus \{\overline{v}\}$
\begin{align}
  &Later\_Disagreeing(p,~!L^H_M(!L^H_J(\sigma')(\overline{v}))(v'), v', \sigma') \\
  &= \{m \in Later\_From(!L^H_M(!L^H_J(\sigma')(\overline{v}))(v'), v', \sigma') : \neg p(Estimate(m)) \} \\
  &= \{m \in \{m' \in \sigma' : Sender(m') = v' \land Later(m', !L^H_M(!L^H_J(\sigma')(\overline{v}))(v'))\} : \neg p(Estimate(m)) \} \\
  &= \{m \in \{m' \in \sigma \cup \{\overline{m}\} : Sender(m') = v' \land Later(m', !L^H_M(!L^H_J(\sigma')(\overline{v}))(v'))\} : \neg p(Estimate(m)) \} \\
  &= \{m \in \big(\{m' \in \sigma : Sender(m') = v' \land Later(m', !L^H_M(!L^H_J(\sigma')(\overline{v}))(v'))\}~ \cup \\
  &~~~~\{m' \in \{\overline{m}\} : Sender(m') = v' \land Later(m', !L^H_M(!L^H_J(\sigma')(\overline{v}))(v'))\}\big) : \neg p(Estimate(m)) \} \\
  &= \{m \in \big(\{m' \in \sigma : Sender(m') = v' \land Later(m', !L^H_M(!L^H_J(\sigma')(\overline{v}))(v'))\}~ \cup \\
  &~~~~\{\overline{m} : Sender(\overline{m}) = v' \land Sender(\overline{m}) \neq v' \land Later(\overline{m}, !L^H_M(!L^H_J(\sigma')(\overline{v}))(v'))\}\big) : \neg p(Estimate(m)) \} \\
  &= \{m \in (\{m' \in \sigma : Sender(m') = v' \land Later(m', !L^H_M(!L^H_J(\sigma')(\overline{v}))(v'))\} \cup \emptyset) : \neg p(Estimate(m)) \} \\
  &= \{m \in \{m' \in \sigma : Sender(m') = v' \land Later(m', !L^H_M(!L^H_J(\sigma')(\overline{v}))(v'))\} : \neg p(Estimate(m)) \} \\
  &= \{m \in Later\_From(!L^H_M(!L^H_J(\sigma')(\overline{v}))(v'), v', \sigma) : \neg p(Estimate(m)) \} \\
  &= Later\_Disagreeing(p,~!L^H_M(!L^H_J(\sigma')(\overline{v}))(v'), v', \sigma) \\
  &= Later\_Disagreeing(p,~!L^H_M(Justification(\overline{m}))(v'), v', \sigma) \\
  &= \emptyset \\ % by above lemmas
\end{align}

\textbf{5:} $\forall v \in V \setminus \{\overline{v}\}, \forall v' \in \{\overline{v}\}, Later\_Disagreeing(p,~!L^H_M(!L^H_J(\sigma')(v))(v'), v', \sigma') = \emptyset$


\begin{align}
&\forall v \in V \setminus \{\overline{v}\}, \forall v' \in \{\overline{v}\}, Later\_Disagreeing(p,~!L^H_M(!L^H_J(\sigma')(v))(v'), v', \sigma') = \emptyset \\
&\iff \forall v \in V \setminus \{\overline{v}\}, Later\_Disagreeing(p,~!L^H_M(!L^H_J(\sigma')(v))(\overline{v}), \overline{v}, \sigma') = \emptyset \\
\end{align}

$\forall v \in V \setminus \{\overline{v}\},$
\begin{align}
&Later\_Disagreeing(p,~!L^H_M(!L^H_J(\sigma')(v))(\overline{v}), \overline{v}, \sigma') \\
&= Later\_Disagreeing(p,~!L^H_M(!L^H_J(\sigma)(v))(\overline{v}), \overline{v}, \sigma')\\
&= \{m \in Later\_From(!L^H_M(!L^H_J(\sigma)(v))(\overline{v}), \overline{v}, \sigma') : \neg p(Estimate(m)) \} \\
&= \{m \in \{m' \in \sigma' : Sender(m') = \overline{v} \land Later(m', !L^H_M(!L^H_J(\sigma)(v))(\overline{v}))\} : \neg p(Estimate(m)) \} \\
&= \{m \in \{m' \in \sigma \cup \{\overline{m}\} : Sender(m') = \overline{v} \land Later(m', !L^H_M(!L^H_J(\sigma)(v))(\overline{v}))\} : \neg p(Estimate(m)) \} \\
&= \{m \in \big(\{m' \in \sigma : Sender(m') = \overline{v} \land Later(m', !L^H_M(!L^H_J(\sigma)(v))(\overline{v}))\}~ \cup \\
&~~~~\{m' \in \{\overline{m}\} : Sender(m') = \overline{v} \land Later(m', !L^H_M(!L^H_J(\sigma)(v))(\overline{v}))\}\big) : \neg p(Estimate(m)) \} \\
&= \{m \in \{m' \in \sigma : Sender(m') = \overline{v} \land Later(m', !L^H_M(!L^H_J(\sigma)(v))(\overline{v}))\} : \neg p(Estimate(m)) \} ~ \cup \\
&~~~~\{m \in \{m' \in \{\overline{m}\} : Sender(m') = \overline{v} \land Later(m', !L^H_M(!L^H_J(\sigma)(v))(\overline{v}))\} : \neg p(Estimate(m)) \} \\
&= \{m \in Later\_From(!L^H_M(!L^H_J(\sigma)(v))(\overline{v}), \overline{v}, \sigma) : \neg p(Estimate(m)) \} ~ \cup \\
&~~~~\{m \in \{m' \in \{\overline{m}\} : Sender(m') = \overline{v} \land Later(m', !L^H_M(!L^H_J(\sigma)(v))(\overline{v}))\} : \neg p(Estimate(m)) \} \\
&= Later\_Disagreeing(p,~!L^H_M(!L^H_J(\sigma)(v))(\overline{v}), \overline{v}, \sigma) ~ \cup\\
&~~~~\{m \in \{m' \in \{\overline{m}\} : Sender(m') = \overline{v} \land Later(m', !L^H_M(!L^H_J(\sigma)(v))(\overline{v}))\} : \neg p(Estimate(m)) \} \\
&= \emptyset \cup \{m \in \{m' \in \{\overline{m}\} : Sender(m') = \overline{v} \land Later(m', !L^H_M(!L^H_J(\sigma)(v))(\overline{v}))\} : \neg p(Estimate(m)) \} \\
&= \{m \in \{m' \in \{\overline{m}\} : Sender(m') = \overline{v} \land Later(m', !L^H_M(!L^H_J(\sigma)(v))(\overline{v}))\} : \neg p(Estimate(m)) \} \\
&= \{m \in \{\overline{m} : Sender(\overline{m}) = \overline{v} \land Later(\overline{m}, !L^H_M(!L^H_J(\sigma)(v))(\overline{v}))\} : \neg p(Estimate(m)) \} \\
&= \{\overline{m} : Sender(\overline{m}) = \overline{v} \land Later(\overline{m}, !L^H_M(!L^H_J(\sigma)(v))(\overline{v})) \land \neg p(Estimate(\overline{m})) \} \\
&= \{\overline{m} : Sender(\overline{m}) = \overline{v} \land Later(\overline{m}, !L^H_M(!L^H_J(\sigma)(v))(\overline{v})) \land \neg p(Estimate(\overline{m})) \land p(Estimate(\overline{m}))\} \\
&= \emptyset \\
\end{align}

These results together with $Clique(1, V \setminus {\overline{v}}, p, \sigma')$ give our desired result:

\begin{align}
Clique(1, V, p, \sigma) \land \overline{v} \in V \land \overline{v} \notin E(\sigma') &\\
\land ~\forall v \in V, Majority^H(V, !L_J(\sigma)(v)) &\implies Clique(1, V, p, \sigma')
\end{align}

\end{proof}



%---------------------------------------------------------------------------------------------------------------------
%---------------------------------------------------------------------------------------------------------------------
%---------------------------------------------------------------------------------------------------------------------
%---------------------------------------------------------------------------------------------------------------------
%---------------------------------------------------------------------------------------------------------------------
%---------------------------------------------------------------------------------------------------------------------
%---------------------------------------------------------------------------------------------------------------------



\subsection{Equivocations from Validators in Clique do not break cliques}

\begin{defn}[One layer clique oracle threshold size]
\begin{align}
Threshold_t&: \mathcal{P}(\mathcal{V}) \times \Sigma_t \to \{True, False\} \\
Threshold_t(V, \sigma) &:\Leftrightarrow Weight(V) > Weight(\mathcal{V})/2 + t - Weight(E(\sigma))
\end{align}
\end{defn}

\begin{lemma}
  $$
    Threshold_t(V, \sigma) \implies \forall v \in V, Majority^H(V, !L^H_J(\sigma)(v))
  $$
\end{lemma}

\begin{proof}
  \begin{align}
            & Threshold_t(V, \sigma) \\
    \implies& Weight(V) > Weight(\mathcal{V})/2 + t - Weight(E(\sigma)) \\
    \implies& Weight(V) > Weight(\mathcal{V})/2 \\
    \implies& \forall v \in V, Weight(V) > Weight(\mathcal{V})/2 \\
    \implies& \forall v \in V, Weight(V) > Weight(\mathcal{V} \setminus E(!L^H_J(\sigma)(v)))/2 \\
    \implies& \forall v \in V, Majority^H(V, !L^H_J(\sigma)(v))) \\
  \end{align}
\end{proof}



\begin{defn}[Clique oracle with $n$ layers]
\begin{align}
Clique\_Oracle_t&: \mathcal{P}(\mathcal{V}) \times \Sigma_t \to \{True, False\} \\
Clique\_Oracle_t(n, V, \sigma, p) &:\Leftrightarrow Clique(n, V \setminus E(\sigma), \sigma, p) \land Threshold_t(V \setminus E(\sigma), \sigma) \\
\end{align}
\end{defn}

We now argue the following:

\begin{lemma}[Clique oracles preserved over minimal transitions from validators not in clique]
  $\forall p \in P_{\mathcal{C}} : Majority\_Driven^H(p),$ $\forall (\sigma, \sigma') \in Minimal_t, \forall V \subseteq \mathcal{V}$, letting $\overline{m} = ~ !\sigma'\setminus\sigma$, and $\overline{v} = Sender(\overline{m})$,
  $$
  \overline{v} \notin V \setminus E(\sigma) \land Clique\_Oracle_t(1, V, \sigma, p) \implies Clique\_Oracle_t(1, V, \sigma', p)
  $$
\end{lemma}

\begin{proof}
\begin{align}
          &\overline{v} \notin V \setminus E(\sigma) \land Clique\_Oracle_t(1, V, \sigma, p) \\
  \implies&\overline{v} \notin V \setminus E(\sigma) \land Clique(1, V \setminus E(\sigma), \sigma, p) \land Threshold_t(V \setminus E(\sigma), \sigma) \\
  \implies&Clique(1, V \setminus E(\sigma'), \sigma', p) \land Threshold_t(V \setminus E(\sigma), \sigma) \\
  \implies&Clique(1, V \setminus E(\sigma'), \sigma', p) \land Threshold_t(V \setminus E(\sigma), \sigma) \\
          &\land \forall v \in \mathcal{V} \setminus \{\overline{v}\}, (v \in E(\sigma) \iff v \in E(\sigma')) \\
  \implies&Clique(1, V \setminus E(\sigma'), \sigma', p) \land Threshold_t(V \setminus E(\sigma), \sigma) \\
          &\land \forall v \in V \setminus E(\sigma), (v \in E(\sigma) \iff v \in E(\sigma')) \\
  \implies&Clique(1, V \setminus E(\sigma'), \sigma', p) \land Threshold_t(V \setminus E(\sigma), \sigma) \\
          &\land V \setminus E(\sigma) = V \setminus E(\sigma') \\
  \implies&Clique(1, V \setminus E(\sigma'), \sigma', p) \land Weight(V \setminus E(\sigma)) > Weight(\mathcal{V})/2 + t - Weight(E(\sigma)) \\
          &\land V \setminus E(\sigma) = V \setminus E(\sigma') \\
  \implies&Clique(1, V \setminus E(\sigma'), \sigma', p) \land Weight(V \setminus E(\sigma')) > Weight(\mathcal{V})/2 + t - Weight(E(\sigma)) \\
  \implies&Clique(1, V \setminus E(\sigma'), \sigma', p) \land Weight(V \setminus E(\sigma')) > Weight(\mathcal{V})/2 + t - Weight(E(\sigma)) \\
          &\land \sigma \subseteq \sigma' \\
  \implies&Clique(1, V \setminus E(\sigma'), \sigma', p) \land Weight(V \setminus E(\sigma')) > Weight(\mathcal{V})/2 + t - Weight(E(\sigma)) \\
          &\land Weight(E(\sigma)) \leq Weight(E(\sigma')) \\
  \implies&Clique(1, V \setminus E(\sigma'), \sigma', p) \land Weight(V \setminus E(\sigma')) > Weight(\mathcal{V})/2 + t - Weight(E(\sigma')) \\
  \implies&Clique(1, V \setminus E(\sigma'), \sigma', p) \land Threshold_t(V \setminus E(\sigma'), \sigma') \\
  \implies&Clique\_Oracle_t(1, V, \sigma', p) \\
\end{align}
\end{proof}

\begin{lemma}[Clique oracles preserved over minimal transitions from non-equivocating validators]
  $\forall p \in P_{\mathcal{C}} : Majority\_Driven^H(p),$ $\forall (\sigma, \sigma') \in Minimal_t, \forall V \subseteq \mathcal{V}$, letting $\overline{m} = ~ !\sigma'\setminus\sigma$, and $\overline{v} = Sender(\overline{m})$,
  $$
  \overline{v} \in V \setminus E(\sigma) \land \overline{v} \notin E(\sigma') \land Clique\_Oracle_t(1, V, \sigma, p) \implies Clique\_Oracle_t(1, V, \sigma', p)
  $$
\end{lemma}

\begin{proof}
\begin{align}
          &\overline{v} \in V \setminus E(\sigma) \land \overline{v} \notin E(\sigma') \land Clique\_Oracle_t(1, V, \sigma, p) \\
  \implies&\overline{v} \in V \setminus E(\sigma) \land \overline{v} \notin E(\sigma') \land Clique(1, V \setminus E(\sigma), \sigma, p) \land Threshold_t(V \setminus E(\sigma), \sigma) \\
  \implies&\overline{v} \in V \setminus E(\sigma) \land \overline{v} \notin E(\sigma') \land Clique(1, V \setminus E(\sigma), \sigma, p) \land Threshold_t(V \setminus E(\sigma), \sigma) \\
          &\land \forall v \in V, Majority^H(V, !L^H_J(\sigma)(v)) \\
  \implies&\overline{v} \in V \setminus E(\sigma) \land \overline{v} \notin E(\sigma') \land Clique(1, V \setminus E(\sigma'), \sigma', p) \land Threshold_t(V \setminus E(\sigma), \sigma) \\
  \implies&Clique(1, V \setminus E(\sigma'), \sigma', p) \land Threshold_t(V \setminus E(\sigma), \sigma) \\
          &\land V \setminus E(\sigma) = V \setminus E(\sigma') \land Weight(E(\sigma)) \leq Weight(E(\sigma')) \\
  \implies&Clique(1, V \setminus E(\sigma'), \sigma', p) \land Threshold_t(V \setminus E(\sigma'), \sigma') \\
  \implies&Clique\_Oracle_t(1, V, \sigma', p) \\
\end{align}
\end{proof}


\begin{lemma}[Clique oracles preserved over minimal transitions from non-equivocating validators]
  $\forall p \in P_{\mathcal{C}} : Majority\_Driven^H(p),$ $\forall (\sigma, \sigma') \in Minimal_t, \forall V \subseteq \mathcal{V}$, letting $\overline{m} = ~ !\sigma'\setminus\sigma$, and $\overline{v} = Sender(\overline{m})$,
  $$
  \overline{v} \in V \setminus E(\sigma) \land \overline{v} \in E(\sigma') \land Clique\_Oracle_t(1, V, \sigma, p) \implies Clique\_Oracle_t(1, V, \sigma', p)
  $$
\end{lemma}

\begin{proof}
\begin{align}
          &\overline{v} \in V \setminus E(\sigma) \land \overline{v} \in E(\sigma') \land Clique\_Oracle_t(1, V, \sigma, p) \\
  \implies&\overline{v} \in V \setminus E(\sigma) \land \overline{v} \in E(\sigma') \land Clique(1, V \setminus E(\sigma) \setminus \{\overline{v}\}, \sigma', p) \land Threshold_t(V \setminus E(\sigma), \sigma) \\
  \implies&\overline{v} \in V \setminus E(\sigma) \land \overline{v} \in E(\sigma') \land Clique(1, V \setminus E(\sigma) \setminus \{\overline{v}\}, \sigma', p) \\
          &\land Weight(V \setminus E(\sigma')) > Weight(\mathcal{V})/2 + t - Weight(E(\sigma)) \\
  \implies&\overline{v} \in V \setminus E(\sigma) \land \overline{v} \in E(\sigma') \land Clique(1, V \setminus E(\sigma) \setminus \{\overline{v}\}, \sigma', p) \\
          &\land Weight(V \setminus E(\sigma)) - W(\overline{v}) > Weight(\mathcal{V})/2 + t - Weight(E(\sigma)) - W(\overline{v}) \\
  \implies&\overline{v} \in V \setminus E(\sigma) \land \overline{v} \in E(\sigma') \land Clique(1, V \setminus E(\sigma) \setminus \{\overline{v}\}, \sigma', p) \\
          &\land Weight(V \setminus E(\sigma)) - W(\overline{v}) > Weight(\mathcal{V})/2 + t - Weight(E(\sigma)) - W(\overline{v}) \\
          &\land V \setminus E(\sigma') = V \setminus E(\sigma) \cup \{\overline{v}\} \\
  \implies&\overline{v} \in V \setminus E(\sigma) \land \overline{v} \in E(\sigma') \land Clique(1, V \setminus E(\sigma) \setminus \{\overline{v}\}, \sigma', p) \\
          &\land Weight(V \setminus E(\sigma')) > Weight(\mathcal{V})/2 + t - Weight(E(\sigma')) \\
  \implies&Clique(1, V \setminus E(\sigma) \setminus \{\overline{v}\}, \sigma', p) \land Threshold_t(V \setminus E(\sigma'), \sigma') \\
  \implies&Clique\_Oracle_t(1, V, \sigma', p) \\
\end{align}
\end{proof}


Thus, we are finially ready a our major result:


\begin{lemma}[Clique oracles preserved over minimal transitions]
  $\forall p \in P_{\mathcal{C}} : Majority\_Driven^H(p),$ $\forall (\sigma, \sigma') \in Minimal_t, \forall V \subseteq \mathcal{V}$, letting $\overline{m} = ~ !\sigma'\setminus\sigma$, and $\overline{v} = Sender(\overline{m})$,
  $$
  Clique\_Oracle_t(1, V, \sigma, p) \implies Clique\_Oracle_t(1, V, \sigma', p)
  $$
\end{lemma}

\begin{proof}
\begin{align}
          &Clique\_Oracle_t(1, V, \sigma, p) \\
  \implies&Clique\_Oracle_t(1, V, \sigma, p) \land (\overline{v} \notin V \setminus E(\sigma) \lor \overline{v} \in V \setminus E(\sigma)) \\
  \implies&Clique\_Oracle_t(1, V, \sigma, p) \land (\overline{v} \not V \setminus E(\sigma) \lor \\
          &(\overline{v} \in V \setminus E(\sigma) \land \overline{v} \in E(\sigma') \lor \overline{v} \in V \setminus E(\sigma) \land \overline{v} \notin E(\sigma')) \\
  \implies&(Clique\_Oracle_t(1, V, \sigma', p) \lor (Clique\_Oracle_t(1, V, \sigma', p) \lor Clique\_Oracle_t(1, V, \sigma', p)) \\
  \implies&(Clique\_Oracle_t(1, V, \sigma', p) \\
\end{align}
\end{proof}

We now show that if there is a clique, then everyone in the clique is agreeing in the current protocol state.
\begin{lemma}
  $\forall p \in P_{\mathcal{C}} : Majority\_Driven^H(p),$ $\forall (\sigma, \sigma') \in Minimal_t, \forall V \subseteq \mathcal{V}$
\begin{align}
Clique(1, V, p, \sigma) &\implies V \subseteq Agreeing(p, \sigma)
\end{align}
\end{lemma}

\begin{proof}
  \begin{align}
            &Clique(1, V, p, \sigma) \\
    \implies&\forall v \in V, V \subseteq Agreeing(p, L^H_J(\sigma)(v)) \\
            &~~~~\land \forall v' \in V, Later\_Disagreeing(p, L^H_M(L^H_J(\sigma)(v))(v'), v', \sigma) = \emptyset \\
    \implies&\forall v \in V, \forall v' \in V, Later\_Disagreeing(p, L^H_M(L^H_J(\sigma)(v))(v'), v', \sigma) = \emptyset \\
    \implies&\forall v \in V, Later\_Disagreeing(p, L^H_M(L^H_J(\sigma)(v))(v), v, \sigma) = \emptyset \\
    \implies&\forall v \in V, Later\_Disagreeing(p, L^H_M(L^H_J(\sigma)(v))(v), v, \sigma) = \emptyset \\
    \implies&\forall v \in V, \{m \in Later\_From(L^H_M(L^H_J(\sigma)(v))(v), v, \sigma) : \neg p(Estimate(m)) \}  = \emptyset \\
    \implies&\forall v \in V, \forall m \in \{m' \in \sigma : Sender(m') = v \land Later(m', L^H_M(L^H_J(\sigma)(v))(v))\}, p(Estimate(m)) \\
    \implies&\forall v \in V, \forall m \in \{m' \in \sigma : Sender(m') = v \land Later(m', L^H_M(L^H_J(\sigma)(v))(v))\}, p(Estimate(m)) \\
    &\land L^H_J(\sigma)(v) = Justification(L^H_M(\sigma)(v)) \\
    \implies&\forall v \in V, \forall m \in \{m' \in \sigma : Sender(m') = v \land Later(m', L^H_M(L^H_J(\sigma)(v))(v))\}, p(Estimate(m)) \\
    &\land L^H_M(L^H_J(\sigma)(v))(v)) \in Justification(L^H_M(\sigma)(v))\\
    \implies&\forall v \in V, \forall m \in \{m' \in \sigma : Sender(m') = v \land Later(m', L^H_M(L^H_J(\sigma)(v))(v))\}, p(Estimate(m)) \\
    &\land Later(L^H_M(\sigma)(v), L^H_M(L^H_J(\sigma)(v))(v)) \\
    \implies&\forall v \in V, \forall m \in \{m' \in \sigma : Sender(m') = v \land Later(m', L^H_M(L^H_J(\sigma)(v))(v))\}, p(Estimate(m)) \\
    &\land Later(L^H_M(\sigma)(v), L^H_M(L^H_J(\sigma)(v))(v)) \land Sender(L^H_M(\sigma)(v)) = v \land L^H_M(\sigma)(v) \in \sigma\\
    \implies&\forall v \in V, \forall m \in \{L^H_M(\sigma)(v)\}, p(Estimate(m)) \\
    \implies&\forall v \in V, p(Estimate(L^H_M(\sigma)(v))) \\
    \implies&\forall v \in V, v \in Agreeing(p, \sigma) \\
    \implies& V \subseteq Agreeing(p, \sigma) \\
  \end{align}
\end{proof}



Now, we will show that if there is a threshold sized clique where no validators are equivocating, then the estimator will agree.

\begin{lemma}
  $\forall p \in P_{\mathcal{C}} : Majority\_Driven^H(p),$ $\forall (\sigma, \sigma') \in Minimal_t, \forall V \subseteq \mathcal{V}$
\begin{align}
Clique(1, V \setminus E(\sigma), p, \sigma) \land Threshold_t(V \setminus E(\sigma), \sigma) &\implies \forall c \in \mathcal{E}(\sigma), p(c)
\end{align}
\end{lemma}

\begin{proof}
  \begin{align}
            &Clique(1, V \setminus E(\sigma), p, \sigma) \land Threshold_t(V \setminus E(\sigma), \sigma) \\
    \implies&V \setminus E(\sigma) \subseteq Agreeing(p, \sigma) \land Threshold_t(V \setminus E(\sigma), \sigma) \\
    \implies&V \setminus E(\sigma) \subseteq Agreeing(p, \sigma) \land Weight(V \setminus E(\sigma)) > Weight(\mathcal{V})/2 + t - Weight(E(\sigma)) \\
    \implies&V \setminus E(\sigma) \subseteq Agreeing(p, \sigma) \land Weight(V \setminus E(\sigma)) > Weight(\mathcal{V})/2 \\
    \implies&V \setminus E(\sigma) \subseteq Agreeing(p, \sigma) \land Weight(V \setminus E(\sigma)) > Weight(\mathcal{V} \setminus E(\sigma))/2 \\
    \implies& Weight(V \setminus E(\sigma)) \leq Weight(Agreeing(p, \sigma)) \land Weight(V \setminus E(\sigma)) > Weight(\mathcal{V} \setminus E(\sigma))/2 \\
    \implies& Weight(Agreeing(p, \sigma)) > Weight(\mathcal{V} \setminus E(\sigma))/2 \\
    \implies& Weight(Agreeing(p, \sigma)) > Weight(\mathcal{V} \setminus E(\sigma))/2 \\
    &\land Majority\_Driven^H(p) \\
    \implies& Weight(Agreeing(p, \sigma)) > Weight(\mathcal{V} \setminus E(\sigma))/2 \\
    &\land [\forall \sigma' \in \Sigma, Weight(Agreeing(p, \sigma')) > Weight(\mathcal{V} \setminus E(\sigma'))/2 \implies \forall c \in \mathcal{E}(\sigma'), p(c)] \\
    \implies& Weight(Agreeing(p, \sigma)) > Weight(\mathcal{V} \setminus E(\sigma))/2 \\
    &\land [Weight(Agreeing(p, \sigma)) > Weight(\mathcal{V} \setminus E(\sigma))/2 \implies \forall c \in \mathcal{E}(\sigma), p(c)] \\
    \implies& \forall c \in \mathcal{E}(\sigma), p(c) \\
  \end{align}
\end{proof}


Now, with induction, we can show that when a $1$ layer clique oracle exists, it exists in all futures.


\begin{lemma}[Cliques exist in all futures]
$\forall p \in P_{\mathcal{C}} : Majority\_Driven^H(p), \forall \sigma \in \Sigma_t, \forall V \subseteq \mathcal{V}$,
$$
Clique\_Oracle_t(1, V, p, \sigma) \implies \forall \sigma' \in Futures_t(\sigma), Clique\_Oracle_t(1, V, p, \sigma')
$$
\end{lemma}

\begin{proof}
This follows clearly from two facts and the axiom of induction (although we leave a formal proof out for now):
1. $forall (\sigma, \sigma') \in Minimal_t, Clique\_Oracle_t(1, V, p, \sigma) \implies Clique\_Oracle_t(1, V, p, \sigma')$
2. $\forall \sigma' \in Futures(\sigma), \exists (\sigma_1, \cdots , \sigma_{n}) \in \Sigma_t^n : \sigma_1 = \sigma \land \sigma_n = \sigma' \land \forall i < n, (\sigma_i, \sigma_{i + 1}) \in Minimal_t$

Thus, the clique oracle holds in all futures!
\end{proof}


Finially, we can show the final result:

\begin{lemma}[Clique oracle is a safety oracle]
$\forall p \in P_{\mathcal{C}} : Majority\_Driven^H(p), \forall \sigma \in \Sigma_t, \forall V \subseteq \mathcal{V}$,
$$
Clique\_Oracle_t(1, V, p, \sigma) \implies Safe_{\mathcal{C},t}(p, \sigma)
$$
\end{lemma}

\begin{proof}
  \begin{align}
    &Clique\_Oracle_t(1, V, p, \sigma) \\
    \implies& \forall \sigma' \in Futures_t(\sigma), Clique\_Oracle_t(1, V, p, \sigma') \\
    \implies& \forall \sigma' \in Futures_t(\sigma), Clique(1, V \setminus E(\sigma'), p, \sigma') \land Threshold_t(V \setminus E(\sigma'), \sigma') \\
    \implies& \forall \sigma' \in Futures_t(\sigma), \forall c \in \mathcal{E}(\sigma'), p(c) ) \\
    \implies& \forall \sigma' \in Futures_t(\sigma), H(p)(\sigma') \\
    \implies& Safe_{\Sigma,t}(H(p), \sigma) \\
    \implies& Safe_{\mathcal{C},t}(p, \sigma) \\
  \end{align}
\end{proof}
